\chapter{Results of the segmentation}\label{app:results}
Here we present the results of segmentations in the case of manual and labelmap sampling. The left image represents the results of the segmentation after a manual sampling and the other image represents the results after a labelmap sampling.

%\section{Axial view}

  \begin{figure}[htb]\centering
  \includegraphics[width=1\textwidth]{Images/Screenshots/axialcomp.png}
  \caption{Axial view of the segmentation with different sampling methods}{Left figure (A) is an axial view of the T1 the target volume to be segmented. Right figure (B) is an axial view of the segmentation's result. Air is represented in pink, skull in white, white matter in yellow, grey matter in blue and cerebrospinal fluid in red.}
  \end{figure}

%\section{Coronal view}

  \begin{figure}[htb]\centering
  \includegraphics[width=0.9\textwidth]{Images/Screenshots/coronalcomp.png}
  \caption{Coronal view of the segmentation with different sampling methods}{Left figure (A) is an axial view of the T1 the target volume to be segmented. Right figure (B) is an axial view of the segmentation's result. Air is represented in pink, skull in white, white matter in yellow, grey matter in blue and cerebrospinal fluid in red.}
  \end{figure}

%\section{Sagittal view}

  \begin{figure}[htb]\centering
  \includegraphics[width=0.9\textwidth]{Images/Screenshots/sagittalcomp.png}{Left figure (A) is an axial view of the T1 the target volume to be segmented. Right figure (B) is an axial view of the segmentation's result. Air is represented in pink, skull in white, white matter in yellow, grey matter in blue and cerebrospinal fluid in red.}
  \caption{Sagittal view of the segmentation with different sampling methods}\label{fig:NC_C_L}
  \end{figure}