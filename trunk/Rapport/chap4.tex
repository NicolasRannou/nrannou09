\chapter{Results and discussion}\label{sec:results}
This chapter will review what has been done and
mentions the main open questions.
\par
Results using different tools
\par
limitations
\par
improvement (in relation to limitations)
%
\section{Results}
%
Tests has been performed to show the utility of the work done. The results are presented below. The testings are comparaisons between the results obtained with the previous workflow and the new one. The results obtained will then be reviewed by a specialist to evaluate the previous and the new segmented datasets.
%
\subsection{Bias correction}
%
Here we get going with the intensities inhomogeinities correction.
%
\subsubsection{Testing process}
%
%we create a mask.
%
%\begin{figure}\centering
%\includegraphics[width=1\textwidth]{Images/Screenshots/biasCorrection.png}
%  \caption{Module created for the bias correction.% On the left panel, there are the parametersto set for the correction. On the top left red box, the 2d mask is displayed. On the top right box, the 3d mask is shown. On the lower left yellow slice, the biased volume is presented. In the lower right slice, the corrected volume is printed
%  }\label{fig:BiasCorrection}
%\end{figure}



%
\subsubsection{Results}
%
%Here are the resultsfdd ddddddddddd dddddddddddd ddddddddddd ddddddddd dddddddddd ddddddddddd dddddddddd dddddddddd ddddddddd dddddddddd ddddddddd dddddddd dddddddd dddddd dddd dddddddd..
%
\par
\begin{figure}\centering
  \includegraphics[width=.95\textwidth]{Images/Screenshots/goodRegistration.png}
  \caption{Registration after bias correction.}\label{fig:goodRegistration}
\end{figure}
%
\par
%more comments ttttttttttttttttttttt ttttttttttttt ttttttttttt tttttttttttt ttttttttttt ttttttttttt tttttttttttttt ttttttttttttt tttttttttttttt tttttttttttt ttttttttttttttt\\
%
\subsubsection{Evaluation of the results}

After the bias correction, the result of the registration clearly appears to be better. We evaluated accuracy of the regisstration. We used the simple approach of the joint histogram.
\par
The joint histogram evalutaion method is basic comparaison. Let $A$ be a matrix of size $W*L$. $W$ will be the intensity range of the first image used for the comparaison. $L$ will be the intensity range of the second image to be coompared. The matrix is initialized to $0$. Each time that in the same position, there is the same intensities in the two images, we add $1$ in the good cell in the matrix.
After the joint histogram creation, the value at the coordinate ${i,j}$ in the matrix is the number of pixel pairs having gray level $i$ at position ${x,y}$.
%

%
\subsection{Class Selection}
%
\subsubsection{Testing process}
%
\subsubsection{Results}

%
\subsubsection{Specialist's point of view}
%

\subsection{Class visualization}
%
\subsubsection{Testing process}
%
\subsubsection{Results}
%
\subsection{Intensity Normalization}
%
\subsubsection{Testing process}
%
\subsubsection{Results}

\subsection{Global Prior estimation}
%
\subsubsection{Testing process}
%
\subsubsection{Results}
As we can see, the results obtained are good and accurate.The time of processing is fast and most of the people are now able to understand these parameters, even they are not familiar with EMS. This is an intuitive way to get an estimation of the GPW parameters.\\
Of course some sfsdfs must be done.
If there is a strong bias in the images we process, the pixels of a same class will have different values. Then, a segmentation based on intensity will provide bad results. Moreover, if the user wants to segment 2 classes which have the same color, just using the spatial information for example, they won't be able to use it properly.Thus, the user must always keep in mind that it is just an estimation and that he has to check if the values are accurate.

%
\subsubsection{Specialist's point of view}


\section{Future work}
\newpage
\section*{Acknowledgements}
\addcontentsline{toc}{chapter}{\numberline{}Acknowledgements}
Ron Kikinis who gave me the opportunity to carry out my intersnship in the SPL.
Sylvain Jaume who supervises me during all my work.
Andryi, Daniel, Steve?

