\chapter{Results and discussion}\label{sec:results}
This chapter presents the impact of our contributions on the segmentation results. We will now discuss the enhanced usability of the module. It will allow to see how the segmentation has changed. We will then suggest some potential improvements for the EM segment module in Slicer. 

%
\section{Results}
%
Here we start with a presentation of different results, using the different contributions. The results obtained will then be reviewed by a expert radiologist to evaluate the segmentation. The expert who accepted to evaluate the results is Dr. KIKINIS, professor of radiology at Harvard Medical School. We work with the same dataset for all the segmentation. We chose the datasets with significant intensity inhomogeneity in order to show the importance of our contributions.
%
\subsection{Original segmentation}
First we present the segmentation obtained without any contribution. From these results, it will show the importance of the contributions done. Second we describe the testing process. Then we will present the results of the segmentation and finally, we present the evaluation of the clinical expert.
\subsubsection{Method}
Thanks to Sonia Pujol, a tutorial for the EM segmentation in Slicer 3 is available at \url{http://www.na-mic.org/Wiki/images/2/2f/AutomaticSegmentation_SoniaPujol_Munich2008.ppt}. We followed it with a biased image. The atlases we used for the segmentation are the one available at \url{http://www.na-mic.org/Wiki/images/b/b7/AutomaticSegmentation.tar.gz}.

\subsubsection{Results}
Figure \ref{fig:NC} shows the result of the segmentation.
  \begin{figure}\centering
  \includegraphics[width=0.95\textwidth]{Images/Screenshots/NC.png}
  \caption{Results of the segmentation without bias field correction}{Left figure (A) is the target volume to be segmented. Right figure (B) is the result of the segmentation. Air is representted in pink, skull in white, WM in yelo, GM in blue and CSF in red.}\label{fig:NC}
  \end{figure}
  
  %
% In addition to the problem of intensities inhomogeinities during the tissue distributions initialization, we did a manual sampling. As presented in section \ref{sec:CDS}, it can have importance during the segmentation.  
  
\subsubsection{Expert's validation}
Based on the assessment of the expert, the grey matter is over estimated. This is also illustrated Figure \ref{fig:biasedunbiased} where the right image shows that grey matter (light blue) is overestimated in the right image (B) in comparison to the left image (A).

\subsubsection{Discussion}
We speculate that the reasons of the poor segmentation are the incorrect registration and the class ditributions which is poorly defined. It leads to inaccurate means and excessively large variances which deteriorate a lot the segmentation.

\subsection{Bias corrected segmentation}
%
Here we present the results of a segmentation using the intensities inhomogeinities correction. We first describe the testing process. Then we present and interpret the results obtained. Finally we report the assessments of an expert.
%
\subsubsection{Method}
Similarly to the previous section, we follow the EM segmentation tutorial with one extra-step: the bias field correction.

%
\subsubsection{Results}
The results of the final segmentation are presented in Figure \ref{fig:C_M_L}. The left image (A) presents the original unbiased image and the right image (B) the segmented image, using a manual sampling. 
%

  \begin{figure}\centering
  \includegraphics[width=0.95\textwidth]{Images/Screenshots/C_M_L.png}
  \caption{Results of the segmentation with bias correction}{Left figure (A) is the target volume to be segmented. Right figure (B) is the result of the segmentation. Air is representted in pink, skull in white, WM in yelo, GM in blue and CSF in red.}\label{fig:C_M_L}
  \end{figure}

%
%More images with a better resolution are in annexes (), to really see the importance of the contribution.
%

\subsubsection{Expert's validation}
The expert pointed  misegmentation in the skull area. Indeed, some voxels in the skull are incorrectly classified as grey matter and white matter. The expert explained that this misclassification is due to partial volume effect. The partial volume effect is caused by imaging voxels containing two different tissues (skull and air in our case). The mesured intensity for that voxel will be an average of the two tissue at that location. Another misclassification occurs in areas where the skull is classified as air. The reason is that air and skull exhibit similar intensities in MR images.

\subsubsection{Discussion}
Based on our understanding of the algorithm this better segmentation can be explained by a better registration as shown section (\ref{biasfieldcorrectionregistration}). Moroever, as demonstrated it in Figure \ref{fig:biasedunbiased}, the tissues have a better distribution, especially the grey matter (light blue) in this case. The misclassifications in the skull could be corrected by a pre-processing step using skull stripping algorithm.
%
%
\subsection{Label map sampling segmentation}
The segmentation  will now let us evaluate the influence of the sampling method on the segmentation. We first describe the testing process. Then we present and interpret the results obtained. Finally an expert evaluates the results.
%
\subsubsection{Method}
Similarly to the previous section, we follow the EM segmentation tutorial . The extra step that we take is to initialize the class distribution using a labelmap sampling.
%
\subsubsection{Results}

The results of the final segmentation are presented in Figure \ref{fig:NC_C_L}. The left image (A) presents the original corrected image and the right image (B) the result of the segmentation. 

  \begin{figure}\centering
  \includegraphics[width=0.95\textwidth]{Images/Screenshots/NC_C_L.png}
  \caption{Results of the segmentation with labelmap sampling}{Left figure (A) is the target volume to be segmented. Right figure (B) is the result of the segmentation. Air is representted in pink, skull in white, WM in yelo, GM in blue and CSF in red.}\label{fig:NC_C_L}
  \end{figure}
 
%
\subsubsection{Expert's validation}
The skull is better segmented. Nevertheless, in cerebulum, the segmentation differs from the previous one. It is underestimated  with the labelmap sampling whereas before it was over estimated. The cerebulum is a specific area of the brain because there is a lot of white and grey matter in a small area. The segmentation is challenging and similarly as the point raised during the validation of the bias field correction segmentation, the results are due to partial volume effects. According to our expert segmentation of this area is a complex task. A solution to improve the segmention the cerebrulum properly, would be to extract the region of interest and to apply a finer algorithm to this area.

\subsubsection{Discussion}
%
This time, the classes are supposed to have a better distribution than in the previous segmentation, as we saw in Figure \ref{fig:intensitynormalization}. Misclassification within the white matter is reduced because of the better class initialization with the labelmap sampling. More screenshots with a better resolution are presented in App.~\ref{app:results} to have a better understanding of the differences between manual and labelmap sampling.

%
%

%

\section{Perspectives}
Our work focused on improving the usability of EM segment and the selection of the optimum parameters. As a future direction, some effort could spend on the optimization of the algorithm. Indeed, the long processing time is a major drawback of the current implementation.
\par
On one side the processing time both for the bias field correction and for the EM segmentation needs to be improved. A number of users reported this issue as the main limitation of the current implementation. We observe that the processing for 255x255x100 volume is approximatly 30 minutes. Towards this goal the code of the bias field correction algorithm needs to be optimized.
\par
Another issue occurs with images where the distributions of two classes have similar means. In this case, the exploration of the parameters is challenging, especially for the global prior weights. A useful contribution could be to represent each class as a gaussian instead of a finite portion of the histogram.


