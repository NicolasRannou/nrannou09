\documentclass[a4paper,11pt]{report}
%
%--------------------   start of the 'preamble'
%
\usepackage{graphicx,amssymb,amstext,amsmath}
\usepackage[includeheadfoot]{geometry}
\usepackage{fancyhdr}
\usepackage{indentfirst}
\usepackage{hyperref}

\geometry{ hmargin=2.3cm, vmargin=1.3cm }
%
\sloppy
%\usepackage{a4wide}
%
%\usepackage{geometry}
%%    homebrew commands -- to save typing
\newcommand\etc{\textsl{etc}}
\newcommand\eg{\textsl{eg.}\ }
\newcommand\etal{\textsl{et al.}}
\newcommand\Quote[1]{\lq\textsl{#1}\rq}
\newcommand\fr[2]{{\textstyle\frac{#1}{#2}}}
\newcommand\miktex{\textsl{MikTeX}}
\newcommand\comp{\textsl{The Companion}}
\newcommand\nss{\textsl{Not so Short}}

%
\hypersetup{
     backref=true,
     pagebackref=true,
     hyperindex=true,
     colorlinks=false,
     breaklinks=true,
     citecolor=black,
     filecolor=black,
     linkcolor=black,
     linkbordercolor= {1 1 1},
     citebordercolor= {1 1 1},
     bookmarks=true,
     pdftitle={Internship Report - Surgical Planning Laboratory},
     pdfauthor={Nicolas Rannou},
     %&pdfsubject={NIMS: Neurobiological Imaging Management System},
     %pdfkeywords={data sharing, database, neurobiological images, open-source, metadata, workflow, ontology}        
}

%---------------------   end of the 'preamble'
%
\begin{document}
%-----------------------------------------------------------

\pagestyle{fancy}
\fancyhead{}
\fancyfoot{}

\fancyfoot[L]{Nicolas RANNOU}
%\fancyfoot[C]{DO NOT COPY OR DISTRIBUTE}
\fancyfoot[R]{Page \thepage}
%\fancyhead[C]{CONFIDENTIAL}
\fancyhead[L]{Training Period Report}
\fancyhead[R]{Surgical Planning Laboratory}
\renewcommand{\headrulewidth}{0.3pt}
\renewcommand{\footrulewidth}{0.3pt}

%Redéfinition du style fancy - plain, utilisé pour les pages de nouveau chapitre
%Le style par défaut est un style plain
\fancypagestyle{plain}{
\fancyhf{}
\renewcommand{\headrulewidth}{0.3pt}
\renewcommand{\footrulewidth}{0.3pt}

 %Définition des headers identiques à une page normale
\fancyfoot[L]{Nicolas RANNOU}
%\fancyfoot[C]{DO NOT COPY OR DISTRIBUTE}
\fancyfoot[R]{Page \thepage}
%\fancyhead[C]{CONFIDENTIAL}
\fancyhead[L]{Training Period Report}
\fancyhead[R]{Surgical Planning Laboratory}
}


\begin{titlepage}

%\addtolength{\topmargin}{-3cm}
%\addtolength{\textheight}{28cm}
%\setlength{\textwidth}{19cm}
%\setlength{\oddsidemargin}{-1cm}
%\setlength{\footskip}{-3cm}
%\addtolength{\textheight}{2cm}
%\addtolength{\hoffset}{-1cm}
%\addtolength{\textwidth}{2cm} 
%\enlargethispage{2cm}
\begin{center}
%\vspace{4in}

\begin{center}
\begin{minipage}[c]{.45\textwidth}\centering
\includegraphics[width=1\textwidth]{Images/Logos/logo_ISEN.jpg}
\begin{small}\textbf{Institut Sup\'erieur de l'Electronique et du Num\'erique}\\Tel. : +33 (0)2.98.03.84.00\\Fax : +33 (0)2.98.03.84.10\\CS 42807 - 29228 BREST Cedex 2 - FRANCE\end{small}
\end{minipage}
\hfill\begin{minipage}[c]{0.3\textwidth} 
\begin{center}\textbf{
\begin{Huge}N5\end{Huge}
\begin{normalsize}\\School Year 2008/2009\\Training Period\end{normalsize}}
\end{center}

\end{minipage}
\end{center}

\bigskip
\bigskip
\bigskip
\bigskip

\hrule
\bigskip

\begin{minipage}{1\textwidth}
\begin{center}\begin{LARGE} 
New Expectation Maximization segmentation workflow in Slicer 3
\end{LARGE}\end{center}
\end{minipage}
\bigskip
\hrule

\bigskip
\bigskip
\bigskip

\begin{small}From April 6th to August 31th \\ 
At \\ 
Surgical Planning Laboratory (SPL)
\end{small}

\includegraphics[width=.3\textwidth]{Images/Logos/logo_spl.jpg}
\begin{small}\\Brigham \& Women's Hospital\\
 Harvard Medical School \\
 75 Francis St. \\
 Boston, MA 02115\end{small}
%\vskip 1in
\end{center}
\bigskip

\vspace{0.5cm}
\begin{tabular}{*{2}ll}
\underline{Supervisors:} &Ron KIKINIS - SPL, Harvard Medical School - kikinis@bwh.harvard.edu\\
&Sylvain JAUME - CSAIL, Massashusetts Insitute of Technology - sylvain@csail.mit.edu\\
\end{tabular}
\vspace{0.5cm}\\
%referring tearchers
\begin{tabular}{*{2}l}
\underline{Referring Teachers:} &Christine CAVARO-MENARD - M2 SIBM - christine.menard@univ-angers.fr\\
&Dominique MARATRAY - ISEN Brest - dominique.maratray@isen.fr
\end{tabular}
\vspace{0.5cm}\\
%Etudiant
\begin{flushright}by Nicolas RANNOU\end{flushright}

\end{titlepage}
%-----------------------------------------------------------
%-----------------------------------------------------------
%Acknowledgment
\section*{Acknowledgment}
\vspace{0.2cm}
\par
First of all, I would like to thank Brian WANDELL who gave me the opportunity to realize this internship at the Stanford University and who has always been here to assist me in my work. I would also like to thank Pamela WIDRIN for her help on the administrative side, especially to obtain the visa.\\
\par
Of course, my thoughts also go to my team of work and specially Jorge PHILLIPS, Robert F. DOUGHERTY and Anthony SHERBONDY for allowing me to work on such an interesting project, for their assistance when I needed, their time to answer to my questions and their understanding about my bad control of English. I learned a lot with them on very different topics. Indeed, I have worked on the creation of a complete MRI images management system, which involved two main fields: the computer science and the medical imaging science. Moreover, working in a such prestigious place is a good way to discover the research's world and to have the possibility to assist to a lot of talks given by researchers from across the world.\\
\par
Then, I would like to thank Eric STINDEL to have accepted to supervise this internship and Dominique MARATRAY for her assistance.\\
\par 
Finally, I would like to thank all the people working in the VISTA lab for their reception and their friendliness. Thanks to them, my time in the laboratory has been a real pleasure and they made me feel good by including me in the team very easily.
%-----------------------------------------------------------
%-----------------------------------------------------------
%Abstract
\section*{Abstract}
\vspace{0.2cm}
\par
The use of systems to manage large amounts of data becomes increasingly important in the field of neuroscience as researchers work on neuroimaging methods such as positron emission tomography (TEP) and functional magnetic resonance imaging (fMRI). Indeed, these techniques have produced an explosion of new findings in human neuroscience associated with an explosion of the amount and the size of the data involved in the researches. Moreover, the necessity to share both data and results of processing with others researchers and laboratories is very important nowadays in order to create efficient collaborations and thus to help to make new findings.
\par
In this context, this document propose an interesting solution through the creation of the Neurobiological Image Management System (NIMS) allowing the management of neurological data. Thus, this project consists on an experimental software platform designed to ease and to improve the sharing, storage, mining and analysis of data coming from scanners in different groups of research. This has been made in a way that can evolve depending on the needs of each thanks to an open-source and flexible approach. The document presents similar projects existing in other research centers, a first version with a standard neurobiological imaging workflow and gives some ideas about the creation of an ontology for a such system.
\vspace{0.3cm}\\
\underline{Keywords:} data sharing, database, neurobiological images, open-source, metadata, workflow, ontology
\vspace{0.5cm}\\
\section*{R\'esum\'e}
\vspace{0.2cm}
\par
L'utilisation de syst\`emes pour prendre en charge d'importantes quantit\'es de donn\'ees devient de plus en plus fr\'equente dans le domaine des neurosciences alors que les chercheurs travaillent sur des m\'ethodes de neuroimagerie telles que la tomographie par \'emission de positions (TEP) et l'imagerie par r\'esonance magn\'etique fonctionnelle (IRMf). En effet, ces techniques ont produit une explosion de nouvelles d\'ecouvertes en neuroscience associ\'ee \`a une explosion de la quantit\'e et de la taille des donn\'ees impliqu\'ees dans les recherches. De plus, la n\'ecessit\'e de partager \`a la fois les donn\'ees et les r\'esultats des traitements avec les autres (chercheurs et laboratoires) est tr\`es importante de nos jours afin de cr\'eer des collaborations efficaces et ainsi d'aider \`a faire de nouvelles d\'ecouvertes.
\par
Dans ce contexte, ce document propose une solution int\'eressante \`a travers la cr\'eation d'un syst\`eme de management d'images neurobiologiques. Ce projet consiste en une plateforme logicielle exp\'erimentale construite pour faciliter et am\'eliorer le partage, le stockage, l'extraction et l'analyse de donn\'ees provenant de scanners dans diff\'erents groupes de recherche. Ceci a \'et\'e fait d'une mani\`ere qui peut \'evoluer en fonction des besoins de chacun gr\^ace \`a une approche open-source et flexible. Le document pr\'esente des projets similaires dans d'autres centres de recherches, ainsi qu'une premi\`ere version de celui-ci avec un flux de travail (workflow) standardis\'e et donne quelques id\'ees concernant la cr\'eation d'une ontologie pour un tel syst\`eme.
\vspace{0.3cm}\\
\underline{Mots cl\'es :} partage de donn\'ees, base de donn\'ees, images neurobiologiques, open-source, metadonn\'ees, flux de travail, ontologie
%-----------------------------------------------------------
%\begin{abstract}
%A couple of sentences on three or four lines to summarise your work.\\ 
%This is a \LaTeX\ template for undergraduate project reports.\\
%Its detailed contents evolve to reflect FAQs.\\
%Expectation-maximization is very popular for segmentation but it can be tricky to understand and to use.
%A full description of the EMS algorithm is done in this report.
%Different methods for fast parameters exploration are described.
%As part of the research, preprocessing methods like MRI bias field correction will be explained.
%The results obtained will be presented.
%Following the new workflow should allow the user to segment more datasets, more accurately.\\
%
%\textbf{\underline{Keywords:}} segmentation, expectation, maximization, correction, bias.\\

%\bigskip

%\begin{center}\textbf{ Resum\'e }\end{center}
%Quelques phrases pour resumer mon travail.\\ 
%C'est un template \LaTeX\ pour les rapport.\\
%Le contenu peut evoluer.\\
%
%\textbf{\underline{Mots cl\'es :}}segmentation, expectation, maximisation, correction, biais.
%

%\bigskip

%\end{abstract}

%-----------------------------------------------------------
\tableofcontents
\addcontentsline{toc}{chapter}{Contents}
\listoffigures
\addcontentsline{toc}{chapter}{List of figures}
%-----------------------------------------------------------
\chapter{Introduction}
%The first chapter of a well-structured report is always an
%introduction, setting the scene with motivation and context (as in
%Sec.~\ref{context}) and then looking ahead to summarise what's in the
%rest of the report (as in Sec.~\ref{intro:contents}). It's the
%bit that readers look at first --- {\em so make sure it hooks them!}
%
%\section{Presentation of the laboratory}\label{intro}
%%
%\subsection{Harvard Medical School}
%Harvard Medical School (HMS) is one of the graduate schools of Harvard University. It is currently ranked first among American medical schools by U.S. News and World Report.\\
%%
%\begin{figure}[!h] % Un peu plus impératif que le h
%\begin{center}
%\includegraphics[width=0.3\textwidth, scale=0.25]{Harvard_shield_Medical.jpg}                                                                                    
%\end{center}
%\caption{Harvard Medical School's shield}
%\end{figure}
%%
%Located in the Longwood Medical Area of the Mission Hill neighborhood of Boston, Massachusetts, H.M.S. is home to more than 1200 students.\\
%The school has a large and distinguished faculty to support its missions of education, research, and clinical care. These faculty hold appointments in the basic science departments on the HMS Quadrangle, and in the clinical departments located in multiple Harvard-affiliated hospitals and institutions in Boston. There are approximately 2,900 full- and part-time voting faculty members consisting of assistant, associate, and full professors, and over 5,000 full or part-time non-voting instructors.
%%
%\subsection{The Brigham \& Women's Hospital}
%Recognized internationally for its excellence in patient care, its outstanding reputation in biomedical research, and its commitment to educating and training physicians, research scientists and other health care professionals, Brigham \& Women's Hospital (BWH) is a 777-bed teaching affiliate of Harvard Medical School located in the heart of Boston's renowned Longwood Medical Area. Along with its modern inpatient facilities, BWH boasts extensive outpatient services and clinics, neighborhood primary care health centers, state-of-the art diagnostic and treatment technologies and research laboratories.
%%
%\begin{figure}[!h] % Un peu plus impératif que le h
%\begin{center}
%\includegraphics[width=0.3\textwidth, scale=0.25]{brigham_and_womens.jpg}                                                                                    
%\end{center}
%\caption{BWH's shield}
%\end{figure}
%%
%\subsection{The Surgical Planning Laboratory}
%The  Surgical Planning Laboratory (SPL) is a research laboratory in the Department of Radiology of Brigham and Women's Hospital. The Core Mission of the SPL is the extraction of medically relevant information from diagnostic imaging data and to concepts of computation and image analysis to new field of biomedical research. The lab collaborates with groups within Brigham and Women's Hospital, with other researchers at the Harvard Medical School, with local universities such as Harvard and MIT, and with clinicians, researchers, and engineers throughout the world.
%%
%\begin{figure}[!h] % Un peu plus impératif que le h
%\begin{center}
%\includegraphics[width=0.3\textwidth, scale=0.25]{logo_spl.jpg}                                                                                    
%\end{center}
%\caption{SPL's shield}
%\end{figure}
%
\section{Context and motivation}\label{intro:context}
%My training...

Nowadays, medical image processing is becoming a major field of research in most of the laboratories. Indeed, because of the increasing complexity of the data they have to deal with,  physicists need something to help. Help must be provided in many different ways. Before the surgery, to etablish a fast and accurate diagnosis. During the surgery to prevent physicians from errors and to help them to proceed to more precise moves. After the surgery, to see if it succeeded, or to follow the pathology of a patient. The informations brought to the physicians by the tools must be accurate, robust and provide a fast feedback.
%
\par
%
In this context of pre and post operation, plenty of work has already been done. Nevertheless, there is still a lot of work to achieve. Regarding data storage and exchange, the increasing among of information leads us to find more approriate methods for the same purpose. Another interesting contribution of computer science to medicine is image segmentation. New methods have to be developped for a better diagnosis, or to detect new pathologies. A lot of segmentation techniques appeared like level-set segmentation, region growing or texture based segmentation. Each one is adapted to a specific problem like vessel segmentation or tumor detection. Another remarkable contribution is the segmentation based on expectation maximization (EM). It is very well suited for brain tissue segmentation. 
%
\par
%
For segmentation of brain MR images, the Surgical Planning Laboratory (SPL) at the Brigham and Women's Hospital, affiliated to Harvard Medical School, has developped an EM algorithm for segmentation. This model assumes that the MR volumes does not exhibit an intensity inhomogeinity. Moreover, the implementation is not widely used so far, because the complexity of the segmentation process. Finally, some steps seem not very accurate. In this report we will present an approach to enhance the segmentation for inhomogeneous MR images, correcting intensity bias field. We will also provide tools to the end-user for an easier and more accurate segmentation process.
%
%BUT -  So far, the expectation maximization segmentation (EMS) module not used by physicists.
%Indeed, the results are highly dependents on chosen parameters and some parameters estimation may be hard to do.
%Making something easy to use and one which one the user has more control appears to be an obligation.\\
%FINALLY - OR LATER - In this report, we will first introduce you to the EMS algorithms, especially the one implemented in Slicer 3.
%Then, the new tools available in this EMS workflow will be presented.
%Finally, the new results obtained with the new segmentation process will be evaluated through some experiments.
%
%
%This is a template for \LaTeX\ project reports in the Department of
%Mathematical Sciences. It shows a good overall structure for the
%printed document, and shows how to construct it with a master file
%(\texttt{report.tex}) plus subsidiary files (\texttt{chap1.tex},
%\dots, \texttt{app1.tex}, \dots, \texttt{biblio.tex}).
%\par
%At the same time, features of the current version of \LaTeX\ (\LaTeXe)
%are illustrated --- such as mathematical expressions, numbering and
%cross-referencing, bibliography and citations, graphics and tables.
%Comparison of the source files with the printer-ready document will
%answer a few FAQs: \Quote{How can I do \dots\ in \LaTeX?}.
%\par
%However, this is {\em not} a textbook on \LaTeX\ --- for that, use the
%\lq\nss\rq\ notes by Oetiker \etal\ \cite{NSS}. They are written for
%novices, and are a pleasure to read. They are available free on-line,
%and are kept up-to-date. The \LaTeX\ book at \textsl{Wikipedia}
%\cite{WL} includes the \nss\ material and is good for reference too.
%Access these via the \LaTeX\ resources page \cite{LAT}.
%\par
%For more advanced features see \eg\lq\comp\rq\ \cite{MG}.
%\par
%Well-meant advice on \LaTeX\ for report-writing and poster-making is
%available\footnote{From  \texttt{bob.johnson@dur.ac.uk}} in room CM315,
%where there are reference copies of both \comp\ \cite{MG} and
%\textsl{The Graphics Companion} \cite{GRM}.
%\par
%Even if you are misguided enough \cite{AC} to prepare your report in
%\textsl{Word}, this template at least exemplifies a good structure ---
%and gives advice about references and help with typography.

%
\section{Contents}\label{intro:contents}
The main body of this report is divided as follows.
\par
Chapter \ref{sec:EM} focuses on the EM segmentation. Background will be reminded and the algorithm used in the SPL will be fully described. We will also present in this chapter the segmentation workflow developped in the SPL and the limitations of the current implementation.
Chapter \ref{sec:contributions} describes our contributions. It explains the solutions brought to enhance the segmentation and to improve the usability of the current framework.
Chapter\ref{sec:results} shows the results obtained. It aslo discussed about the results obtained. An expert radiologist evaluates the segmentation. Finally, the work which still has to be done in order to enhance the segmention is briefly describe.
%
%The main body of this report is divided as follows.
%\par
%Chap.~\ref{sec:formulas} has some examples of mathematics, then
%Chap.~\ref{sec:graphics} deals with graphics and includes
%Sec.~\ref{sec:tables} about tables. The Conclusion, in
%Chap.~\ref{andfinally}, summarises what's been achieved, the open
%questions and what could be done next.
%\par
%Then comes the Bibliography, listing all sources of material, data and
%computer programs used, \etc. Its construction is explained in
%\cite[Sec.~4.2]{NSS} and there's more about it in App.~\ref{app:refs}.
%\par
%Otherwise appendices typically hold basic background theory, or
%additional or similar examples, or longer proofs (App.~\ref{app:proofs})
% --- anything you need but which would hold up the main flow of the
%story. You could also use an appendix for listings of any computer
%programs that you've written (App.~\ref{app:programs}).
%\par
%Here there's information about using a PC (App.~\ref{app:pc}) plus
%brief advice on grammar and typography (App.~\ref{app:typo}).

\chapter{Expectation-maximization applied to brain segmentation}\label{sec:EM}
%Each main chapter or section should start with a short description of
%what it holds, and why. Top tip --- begin the whole writing enterprise
%with a first draft of this little bit for each chapter. It will force
%you to think about overall structure.\\
%dsfdfs
%\par
%
Here we get going with theory of the expectation-maximization, applied to brain segmentation and show firstly a simple approach of the problem. Then there's a
more realistic approach of the problem with different constraints, followed by a presentation of the algorithm used in Slicer 3\footnote{open source software developped in the SPL for biomedical engineering purpose}.
%
\section{Presentation of the EM segmentation}
%Magnetic resonance imaging (MRI) is very well suited for analyzing human soft tissue anatomy. It provides high resolution high resolution 3D volumetric data with high resolution between soft tissues. Nevertheless, this technique has some disavantages. Indeed MRI images can be alterated by some artifacts as movement, magnetic suceptibility, http://www.e-mri.org, aliasing, etc.. Another main problem is the apparition of a bias on MRIs. This bias results from QDSFQSDF. Correcting this problem is very important in the purpose of image processing. If we don't, the same tissue will have different intensities through the volume, which can bree mistakes during the segmentation process.
The EM algorithm was explained in 1977 by Arthur Dempster, Nan Laird, and Donald Rubin\cite{1}. They generalized and developped a method used in several times by authors, for particular applications. It is widely used to solve problems where data aisre "missing".
The EM algorithm is an iterative algorithm which works in two steps: Expectation and Maximization. It can be use to solve a lot of image processing's problems like classification, restoration\cite{3}, motion estimation\cite{2}, etc.. 
Since the generalization of the algorithm, a lot of related papers were proposed. Most of them bring algorithms derived from the original one to adapt it to particuliar problems using additional informations.
Nowadays, EM algorithms are become a popular tool for classification problems.  It is particulary well suited for brain MR images segmentation.
A lot of algorithms exist. They present complex frameworks using spatial information, neighborhood or intensity inhomogeneities to enhance the classification.\\
In the SPL, the algorithm developped uses spatial, structural and intensity inhomogeneities informations to segment the brain. 
%

\section{Fundamentals}
you should read this part if..
\subsection{Statistical model used for the brain}
\subsection{Gaussian mixture model}
\subsection{Maximum likelihood model}
Here's an inline formula: \(E=mc^2\), and here's the same
thing displayed:\[E=mc^2.\]A matrix needs to be displayed:
\[  \det\left(\begin{array}{cc}
        a & b \\  c & d \\
      \end{array}\right) = ad-bc,   \]
and here's a result that's displayed and labelled:
\begin{equation}\label{eq:display}
\lim_{a\to\infty}\int_0^a\exp(-x^2)\,dx=\fr12\sqrt\pi.
\end{equation}Note the punctuation in eq.~(\ref{eq:display}) \etc,
which recognises that displayed equations are parts of sentences.
\par
Note also that a smaller (\verb+\textstyle+) size of numerical fraction
is often useful in a displayed equation. The appropriate
\verb+\newcommand+ called \verb+\fr+ is defined in the
\Quote{preamble}, in file \texttt{report.tex}.
\par
More complicated limits on \verb+\int+ (and \verb+\sum+) need to
be enclosed in braces $\{\cdots\}$.
\par
In an inline formula, avoid ugly fractions such as \(v=\frac st\) and
\(\frac{dy}{dx}=x+y\). It looks much better to write \(v=s/t\) and
\(dy/dx=x+y\). That is, keep \verb+\frac+ for
display:\[\frac{dy}{dx}=x+y\]where it belongs.
\par
Beware of inadvertent blank lines in your \texttt{.tex} file; they often
creep in immediately after a displayed equation. A blank line gives a
new paragraph, usually indented, which may not be what you want. You can 
avoid blank lines altogether by breaking paragraphs with \verb+\par+
instead, as done here.
%
\section{Expectation maximization algorithm}
first simplified then real
\subsubsection{Simplified algorithm}
\subsubsection{Expectation maximization algorithm}
Now an equation on several lines using \verb+eqnarray+:
\begin{eqnarray*}
  |\vec A|^2 &=& a_1^2+a_2^2  \\
             &=& \sin^2\theta+\cos^2\theta \\
             &=& 1
\end{eqnarray*}--- where the star on \verb+eqnarray*+ suppresses
all line-numbering\footnote{And re-phrase material where any displayed
equation splits between pages.}.
\par
Again, with just one line numbered for reference:
\begin{eqnarray}
  |\vec A|^2 &=& a_1^2+a_2^2  \label{line-one}\\
             &=& \sin^2\theta+\cos^2\theta \nonumber \\
             &=& 1  \nonumber
\end{eqnarray}--- where \verb+\nonumber+ suppresses numbering
otherwise. Now you can refer to equation~(\ref{line-one}).
\par
If you want to number the equations in Chapter 2 as (2.1), (2.2), etc,
then\footnote{Top tip: easily find out about any \LaTeX\ command by
putting it \textit{with} its backslash (and perhaps the word \lq latex')
into your favourite search engine.} put \lq\verb+\numberwithin+' into
\textsl{Google} or see \cite[Sec 8.2.14]{MG}.
\par
Notice that in equations you use \verb+\lim+, \verb+\exp+, \verb+\sin+
and \verb+\cos+ and not plain $lim$, $exp$, $sin$ and $cos$. See
\cite[Sec.~3.3]{NSS} for a list.
\par
For a lot of complicated multi-line formulas it's better to use the more
sophisticated display environments provided by the package
\texttt{amsmath} --- see Sec.~8.2 of \comp\ \cite{MG}.
For instance the \texttt{cases} environment is used to get
\[\theta(x)=\begin{cases}0&\text{if $x<0$,}\\
		1&\text{if $x>0$.}\end{cases}\]
And if you need continued fractions, instead of 
\[x=\frac{1}{a_2+\frac{1}{a_3+\frac{1}{a_4+\cdots}}},\]a better-looking
result
is\[x=\frac{\strut1}{\displaystyle a_2+\frac{\strut1}{\displaystyle
a_3+\frac{\strut1}{\displaystyle a_4+\cdots}}},\]for which
\texttt{amsmath}
provides the command \verb+\cfrac+ \cite[Sec.~8.4.2]{MG}.
\par
The package \texttt{amssymb} \cite[Chap.~8]{MG} extends the range of
mathematical symbols (\eg\(\mathbb{R}\), \(\mathbb{Z}\) and
\(\mathbb{N}\) are available).
\par
And its sister \texttt{amstext} provides the command \verb+\text+ to
put words into a displayed equation \[ \text{like this: }E=mc^2. \]Note
that you need to put in by hand the spacing between the text and the
mathematical symbols.
%
\section{Expectation maximization algorithm used in Slicer 3}\label{angels}
In this part, we present the algorithm which has been intergated in Slicer 3, and discuss of the limitations of this one. Some informations has been added to the algorithm to get the best and most automatic segmentation as possible.
%
\subsection{Spatial information}
%
\subsection{Structure information}
%
\subsection{Intensity inhomogeneities information}
%
\subsubsection{Discussion}
%This is theorem~\ref{angels}.
%\begin{description}
%\item[Theorem]: The whole is greater than the sum of its parts.
%\item[Proof:] See App.~\ref{app:proofs}, Sec.~\ref{pf:angels}.
%\end{description}
%For numbered theorems, use \verb+\newtheorem+ \cite[Sec.~3.8]{NSS}.
%\par
%Here's the main sequence of theorems.
%\newtheorem{Main}{Theorem} % this could go into the preamble
%\begin{Main}[my first result]
%The first theorem, showing
%\[ \bar D^n\quad\text{and}\quad\bar{D^n} \]
%where the first bar is over just the \lq D\rq\ and the second is over
%the whole expression. Also the word \lq and\rq\ is inside displayed
%maths, using \verb+\text+ with extra space before and after.
%\end{Main}And that's how to do left and right quotation marks, too,
%with \verb+\lq+ and \verb+\rq+.
%\begin{Main}
%The second theorem --- with \verb+\overline+ inline: $\overline{D^n}$.
%\end{Main}
%\begin{Main} the third theorem\end{Main}
%Here's another sequence of results.
%\newtheorem{second}{Lemma} % this too could go into the preamble
%\begin{second}[my second result]
%The next result (a mere lemma) \dots\end{second}
%\begin{second}\dots\ and the next\end{second}
%\comp\ \cite[Sec.~3.3.3]{MG} describes how to use the
%packages \texttt{amsmath} and \texttt{amsthm} to state (and
%cross-reference) theorems, lemmas, definitions, proofs, \etc\ in a more 
%sophisticated way.
%
\section{Nothing}
Each chapter should end with a round-up of its contents and a link
with the contents of the next.
\par
After formulas, equations and theorems, the next important
topics are graphics and tables.

\chapter{Contributions}\label{sec:contributions}
In this chapter we will present all the contribution brought to enhance the segmentation workflow in Slicer 3. We propose solution to the problem cited in the previous chapter (\ref{su:limitations}). We propose solutions to enhance the class selection method and to allow the user to evaluate his selection. We deal with the registration problem in a third step. The Finally, we present some tools we added to help the user to find the good intensity normalization value and an estimation of the hierarchical parameters.
%
\section{Class Distribution selection}\label{sec:CDS}
%
During parameters initialization, the user has to define each class distribution. The previous method of selection presents some limitations and we propose a new approach.
%
\subsection{Interest}
%
So far, the user has two choices to define each class distribution. 
\par
The first possiblity consists in entering manually the intensities mean value and variance for each class, for each volume to be processed. This way, the user can be very precise and accurate when he defines each class. But it is very hard to find the good mean value and variance for each class for each volume. Morever, each time we want to process a new volume, we have to redefine mean values and variances. It is not convenient and it can a lot of time to find accurate values for the parameter initialization. 
\par
The next approch consisted in defining a class model by manual sampling. For each class, the user clics in the related part of the volume. The problem with this method is that you compute your mean value and variance using only a few samples. Your sample will never be bigger that one hundred points because it is not convenient. Then, your mean values and variances are not accurate. Moreover, results are not reproducible with this method. This the number of samples is reduced, means and variances can vary a lot with one more sample and you can never reproduce two times the same initialization.
\par
Because of all these limitations,  we proposed a new approach using a label map, to estimate each class model.
%
\subsection{Method used}
%
The idea is to create a label map. This map contains colors. There is one color for each class we want to segment. The relation color/class is stored in parameter $H$, in the EM algorithm. This relation color/class is set up during the tree creation step (section  \ref{GUI}).
\par
The user creates a label map by coloring caracterisc regions for each tissue to segment, in the appropriate color. This gives a spatial information to the algorithm. It can now estimate automaticly the mean value and covariance of each class, for each tissue, using this label map. The reason why covariance matrix is estimated, instead of the simple variance is presented in the next section.
\par 
It is very convenient because, since the algorithm needs a good initialization, we can easily define a sample of hundred of points for each class. The results will representatives. Moreover, the results are now reproducible. Indeed, we can store then re-use the same label map. The results will remain the same.

Figure (\ref{fig:labelmaps}) represent an example of a label map. (A) represent an axial view of the vloume to be segmented and (B) the label map created for this volume. Each color represents a tissue to be segmented.

\begin{figure}\centering
  \includegraphics[width=0.9\textwidth]{Images/Screenshots/labelmaps.png}
  \caption{Axial view of the label map.}\label{fig:labelmaps}
\end{figure}
\subsection{Evaluation}
To estimate the influence of the contribution, we processed to a simple comparaison. We selected manually ten points of the white matter as fine as possible. With this sample, we estimate mean value and covariance matrix, through 2 volumes. The second step of the evaluation consisted in estimating the same values for the same tissue using a label map (sample bigger than 200 points). \\ \\
\textbf{Results:}

\begin{equation*}
\mu_{Manual1} =  543 \mbox{~~,~~} \mu_{Manual2} =  93 \mbox{~~and~~} \mathbf{\Sigma_{Manual}} = 
 \begin{bmatrix}
   1105 & -25 \\
   -25 & 1308
 \end{bmatrix}
\end{equation*}

\begin{equation*}
\mu_{Label1} =  489 \mbox{~~,~~} \mu_{Label2} =  92 \mbox{~~and~~} \mathbf{\Sigma_{Label}} = 
 \begin{bmatrix}
   592 & -201 \\
   -201 & 280
 \end{bmatrix}
\end{equation*}

\par
The mean values ($\mu_{Manual}$ and $\mu_{Labell}$) differs slightly and covariance matrices ($\Sigma_{Manual}$ and $\Sigma_{Label}$) differ significantly. It means that our approach is usefull and it shows the importance of having a large sample to evaluate means and covariance values in a such process. The square variance of the class, is in position $(1,1)$ in the covariance matrix for the first volume. For the second volume, it is in position $(2,2)$. Variance expresses the range through which the class is expected to be. The estimation of this interval is way more precise with label map than with manual sampling. We can explain it with the important number of samples used for the estimation.

%
\section{Class Distribution visualization}\label{sec:tables}

An important contribution is a tool which allows to visualize the distribution of the classes to be segmented.
\subsection{Interest}
As discussed before, the algorithm is sensible to the initalization. It means that the initialization has to be good. Once the parameters are chosen, the user has no tools to know if his selection is accurate. Two classes to segment can't have too close means and variances. Even if the user sees the values he chooses, it is not easy to know if two classes to be segmented are too similar or not.
\subsection{Our approach}
The objective is to provide the user the most accurate and usefull vizualisation as possible.
\par
We first assumed that each class has a normal distribution. We first decided to plot the gaussian in 3D, using the multivariate normal distribution.
In the 2-dimensional nonsingular case, the probability density function is 

\begin{equation}\label{GDDDPGDGS}
f(x,y)=\frac{1}{2 \pi \sigma_y \sigma_y \sqrt{1-\rho^2}} \operatorname*{exp}\Big ( -\frac{1}{2(1-\rho^2)}\Big( \frac{(x-\mu_x)^2}{\sigma_x^2} + \frac{(y-\mu_y)^2}{\sigma_y^2} - \frac{2\rho(x-\mu_x)(y-\mu_y)}{\sigma_x \sigma_y}\Big)\Big)   
\end{equation}

(see \cite{15}). $x$ and $y$ are the position of the pixel in the 2D space. $f(x,y)$ will return the value (heigth) of the $(x,y)$ pixel. Each $X$ and  $Y$ axe represent one volume. Let's first say that the range of the $X$ and $Y$ axes are the intensity range of the $X$ and $Y$ volumes. $\mu_x$ is the mean value of the class in the $X$ volume. $\mu_y$ is the mean value of the class in the $Y$ volume. $\sigma_x$ and  $\sigma_y$ are the variance of the tissue in its respective volume. $\rho$ is the correlation between $X$ and $Y$. It indicates the strength and direction of a linear relationship between two random variables ( see \cite{16}). 


\par
We can easily deduce $\rho$ from the covariance matrix $\Sigma$ (see \cite{17}). Indeed, in the 2 dimensional case, the covariance matrix can be expressed as

\begin{equation*}
\mathbf{\Sigma} = 
 \begin{bmatrix}
   \sigma_x^2 & \rho \sigma_x \sigma_y \\
   \rho \sigma_x \sigma_y & \sigma_y^2
 \end{bmatrix}
\end{equation*}

The covariance matrix and the mean values for each class to segment for each image are computed during the labelmap sampling (section~\ref{sec:CDS}). 
\par
Please note that if $\rho = 1$, it means that the two radom variables are the same and we can't use the same density probability function anymore. We use the classic normal distribution's formulation
\begin{equation}\label{GDPGDGS}
f(x,y)=A\operatorname*{exp}-\Big ( \frac{x-\mu_x}{2\sigma_X^2} + \frac{y-\mu_y}{2\sigma_Y^2}\Big)   
\end{equation}

$A$ is the amplitude of the gaussian.
\par
We said that the range of the $X$ and $Y$ axes are the intensity range of the $X$ and $Y$ volumes. The problem with this approach appears if the classes to segment are not spread over all the intensities. Indeed, the vizualisation is not good then: the gaussian is only localized in a small portion of the 2D plane. We want to "zoom" on the region of interest. We decided to change the range of the two axes. Then range will be re-defined for each image. Let's present it for a given image $X$. It is now the difference between $Max$, the maximum value extracted with the label map, between all the samples between all the classes for the image $X$, and $Min$, its opposite.
\par
For our particular purpose of tissues' distributions' vizualisation we didn't use exactly the formulas (\ref{GDDDPGDGS}) and (\ref{GDPGDGS}). Indeed, we don't normalize the curves: we set $\frac{1}{2 \pi \sigma_y \sigma_y \sqrt{1-\rho^2}}$ to $1$ in (\ref{GDDDPGDGS}) and $A$ to $1$ in (\ref{GDPGDGS}). We do it because we have no clue about the importance of each class so we don't want to "disavantage" any one. A compromise could be to have an amplitude factor proportional to the number of pixels which are supposed to constitute the class.
\par 
We finally obtain the results figure (\ref{fig:intensitynormalization}) for different sampling methods.
\begin{figure}\centering\label{classdistribution}
  \includegraphics[width=0.9\textwidth]{Images/Screenshots/classdistribution.png}
  \caption{Distribution of the class to be segmented}\label{fig:intensitynormalization}
 \end{figure}
 
(A) present the visualization we obtain after a manual sampling and (B) the results after a label map sampling. The distributions slightly differ. The dark blue point, on the left corner represents the air. The skull is represented in purple, the white matter in yellow, the gray matter in green and the cerebrospinal fluid (CSF) in red. The X axis represents the T1 volume and the Y axis the T2 volume. On (A), the skull is significantly more spreaded than in (B), especially along the X axe (T1). According to an expert, this result is better in the case of labelmap sampling, indeed, the skull should almost have the same variance through T1 and T2, what is almost the case in (B). Moreover, regarding the CSF and the grey matter, the distribution is better in (B) again. According to the same expert, the intensities of CSF and grey matter, in the T1 volume are close and can't be as distant as in (A). These two observations show the utility of the labelmap sampling in order to have accurate tissue distributions.


\section{MRI Bias Field correction}\label{biasfieldcorrectionregistration}

The registration step could present some problems if the image to segment has intensities inhomogeinities. Moreover, bias field canal present problems regarding the classes distribution. We will first remind the problems, then we present the solutions proposed.

\subsection{Interest}

In the segmentation process, a registration step is required. Registration consists in finding a trasformation to fit two images as well as possible. The main methods are described in \cite{21}. Only one pre-processing (intensity normalization) is done before the registration. The problem is that the algorithm is designed to treat MR images. MR images are often corrupted by a bias field. Thus, the image to register present intensities inhomogeinities. These inhomogeinities can deteriore a lot the registration.
\par
On figure (\ref{fig:bfexemple}), we present the result of the registration between an atlas and a biased MR image. Note that the target MR image has been normalized to have the same mean value as the atlas. The results is clearly bad. A solution must be brought to enhance this step and so the segmentation.

  \begin{figure}[ht]\centering
  \includegraphics[width=0.5\textwidth]{Images/Screenshots/badRegistration.png}
  \caption{Result of registration of a biased MR image without correction}\label{fig:bfexemple}
  \end{figure}
  
\par
Moreover, this bias field lead to another main issue. Indeed, if the MR image is biased, a the intensity of a given tissue will vary a lot through the volume, even if it should not. Then the algorithm will be initialized with wrong mean and variance values and the segmentation won't be as good as it would be with a corrected image.
  
\subsection{Our approach}

The idea simply consists in correcting the bias field of the MR image before this step. Thus, the registration will be significantly enhanced.Since the registration is better, it should also increase the segmentation.
\par
To correct the bias field, we used the non-parametric approach presented by Sled \textit{et al} in \cite{19}. We choose a non-parametric approach because it doesn't require prior information like the number of tissues to correct or the mean value of each tissue to be corrected. We implement an ITK\footnote{open-source C++ toolkit for segmentation and registration. See \cite{13}.} filter (\cite{14}) in Slicer 3. 
\par
We can describe the new segmentation workflow in Slicer 3 as we do in figure (\ref{fig:wfwbc}).

\begin{figure}[ht]\centering
  \includegraphics[width=1\textwidth]{Images/Graphics/newalgo.png}
  \caption{New algorithm pipeline}\label{fig:wfwbc}
  \end{figure}

\par
We choose not to implement it in Slicer 3 as part as the EM Segment module. Indeed, users may want to correct the bias field in MR images for other purposes. Moreover, because it would be the first pre-processing step, it is possible to do so. The user will first have to correct the intensities inhomogeinities via the module then use the corrected images in the EM segmentation module.

  
  
\par
After the bias correction, we obtain interesting results (\ref{fig:goodRegistration}). (A) represents the atlas to be registered. (B) represents the target volume for the registration. (C) repre sents the atlas after the registration. The result of the registration visually appears to be better but we can't be satisfied of this visual verification. We need a formal evaluation method.

\begin{figure}\centering
  \includegraphics[width=.5\textwidth]{Images/Screenshots/goodRegistration.png}
  \caption{Registration after bias correction.}\label{fig:goodRegistration}
\end{figure}

\subsection{Evaluation}

We evaluated accuracy of the registration using the joint histograms method.
The joint histogram evaluation method is basic comparaison between two images. Let $A$ be a matrix of size $W*L$. $W$ will be the intensity range of the first image used for the comparaison. $L$ will be the intensity range of the second image to be compared. The matrix is initialized to $0$. Each time that in the same position, there is the same intensities in the two images, we add $1$ in the corresponding cell in the matrix. Thus, a perfect registration, would lead to an array of zeros, expect on the diagonal.
After the joint histogram creation, the value at the coordinate ${i,j}$ in the matrix is the number of pixel pairs having gray level $i$ at position ${x,y}$.
\par
In figure (\ref{fig:joint1}), (A) and (B) compares the joint-histogram of the biased image and its atlas, respectively before and after registration.  The color scale used is the following one: if there are a lot of pixels in a cell of the array, the cell will be displayed in red. We the number of points decreases continously. Red becomes orange, yellow, green then blue.

\begin{figure}\centering
  \includegraphics[width=.9\textwidth]{Images/Screenshots/jointhistograms1.png}
  \caption{Joint histograms to evaluate the registration.}\label{fig:joint1}
\end{figure}


In figure (\ref{fig:joint2}), (A) and (B) compares the joint-histograms of the corrected image and its atlas, before (A) and after (B) registration.

\begin{figure}\centering
  \includegraphics[width=.9\textwidth]{Images/Screenshots/jointhistograms2.png}
  \caption{Joint histograms to evaluate the registration.}\label{fig:joint2}
\end{figure}

It clearly appears the the difference between before and after registration is more significant if the bias field is corrected. Indeed, we compare the biased joint histogram, there is no significative difference between before and after registration. That means that the registration didn't improve the similarity between the images. On the contrary, if we correct the bias field, it the number of point around the matix's diagonal clearly increases. It means that now, the images are more similiars than before. It shows the utility of the contribution. The influence of these results after the whole segmentation process will be presented in the next chapter.
\par
Regarding the tissue intensity distribution effect, we will present some results using the tool we developped in the previous section, to evaluate the classes' distribution.
Using the same labelmap for sampling, we obtain two totaly different distributions. Figure (\ref{fig:biasedunbiased}) clearly show the two different distributions. The relation tissue/color is the same as the one in figure (\ref{fig:intensitynormalization}). (A) presents the distribution before bias correction. (B) presents the distribution through the corrected volume.

\begin{figure}\centering
  \includegraphics[width=.9\textwidth]{Images/Screenshots/biasedunbiased.png}
  \caption{Tissue distributions.}\label{fig:biasedunbiased}
\end{figure}
The tissues variances are huge, especially for the CSF (red) through T2. Its distribution through the T2 volume is huge. According to an expert, it should clearly not. It should only be in the high intensities of the T2 volume. Our result means that the CSF class contents high and low intensities. It is due to the bias and is not acceptable.
\par
Intuitively, we understand that the segmentation process should be a lot deteriored because of the two issues we presented (registration and distribution).
%\subsection{Results}
%corrected not corrected
%parameters explanation
\subsection{Registration parameters}
Even if we are doing a non-parametric registration, some parameters has to be defined. "Non-parametric" means no information about the volume and classes to correct. We will first present and explain you the parameters. In a second step, we will propose you some parameters adaptated to different problems.

\begin{itemize}
\item \textbf{Shrink factor}\\
\hspace*{4 mm} It is a factor which is used to reduce the size of the image to be processed. A down-sampling is done by the bias correction filter.

\item \textbf{Maximum Number Of Iterations}\\
\hspace*{4 mm} Optimization of the bias field occurs iterativly until the number of iterations exceeds the maximum specified by this variable.

\item \textbf{Bias Field Full Width At Maximum Iteration}\\
\hspace*{4 mm} The bias field is modelled with a Gaussian. This variable characterizes this Gaussian (see \cite{20}) and can be presented as a parameter which defines the strength of the bias.
\end{itemize}

\par
From the understanding of the parameters, it is now obvious that if you want to increase the time of processing, you should increase the shrink factor or reduce the maximum number of iteration. The limitation is that is can deteriore the bias field correction. You can also increase or reduce the Bias field full width at maximum iteration, depending on the importance of the bias.
%


\section{Intensity Normalization}\label{intensitynormalizationddd}
Another very usefull contribution is a tool which helps the user to determine the good normalization value.
\subsection{Interest}
As discussed in section (\ref{GUI}), at the step 4, an intensity normalization is done. Will already presented the utility of an intensity normalization in the same section. The problem is that the user has no tools to find the good values for the segmentation. He has to guess the mean intensity of the voxels in the MR image, background exluded. This is of course not doingable in practise.
\subsection{Our approach}
We implemented a simple tool, to allow the user to find easily and accurately this normalization value.
\par
The first step of the work consited in creating the histogram associated to the image. The $Y$ axe which presents the number of pixels for each intensity in the volume uses a log-scale because the range is huge. The log scale reduces considerably the range. We then added a cursor in this histogram. Using it, the user can choose the intensity which will be the "limit" between the background and part of interest. Finally, while the cursor is moving, our algorithm computes automatically the mean value of the voxels in the volume, from this intensity range to the highest intensity range. This is the normalization value.
\par
We present in figure (\ref{fig:intensitynormalization2}) the tool we developped. A T1 volume has been loaded. The user can move the cursor in the histogram. While moving, it returns in real-time the normalization value for the given position of the cursor in the lower frame.

\begin{figure}\centering
  \includegraphics[width=0.5\textwidth]{Images/Screenshots/intensityNormalization.png}
  \caption{Tool developped for the intensity normalization parameter estimation}\label{fig:intensitynormalization2}
\end{figure}
%


\section{Global Prior Weights Estimation}\label{GPSPDGSPGS}

The last contribution to the EMS is a tool which provides the user an easy and fast way to estimate the approximate size (also called Global Prior Weights (GPW)) of each class to be segmented.

\subsection{Presentation of the problem}
During the segmentation process,at the $6^{th}$ step of the intialization (section \ref{GUI}), you have to provide to the algorithm an estimation of the size of each tissue to be segmented. 
First of all if there are a lot of structures to segment, they user can spend a lot of time during this step. Moreover, the user may not know at all which size to choose. A tool to estimate of the good sizes to choose is needed. 
%We must also keep in mind that the end users are physicists. They might don't understand what the parameters meanings and providing them a visual feedback could help them a lot.
\subsection{Our approach}
We divided the problem in two parts. The first part will be about providing the user a real-time feedback regarding the global prior weights estimation.
The second part will consist in developping an algorithm which fills automatically the tree.
%\subsubsection{Fast user feedback}
%We can divide the feedback part in 3 steps: the histogram computation and utilisation, the multicolumn list and the labelmap generated.
%The histogram allows the user to manual segment classes based on intensity.
%The multicolumn list allows the user to change the order of the classes in the histogram.
%The labelmap provides to the user a visual feedback, base on the segmentation realized in the histogram.
%Using these three complementary tools, the user, even if he is not initiated can estimate easily, accurately and rapidely the GWP.
%\\schema
%\subsubsection{Global prior weights evaluation}
%The algorithm used to estimate the weight of each node is iterative. It starts from the root and goes to the leaves. It evaluates the weight of the childs of the active node at each iteration. Here is a description of the algorithms used to compute the GPW of each node.\\
%DEscirption algo 1\\\\
%
%
%\begin{minipage}{1\textwidth}
%
%\hrule
%\textbf{\\Algorithm 1:} \textsc{TreeWeightEstimation}(R, W)
%\hrule
%variables definition
%\textbf{\\define}  C = CHILD(R) $\leftarrow$ set of childrens of root R\\ 
%\textbf{define}  LEAF(C)      $\leftarrow$ set of leaves of tree with roots C\\ 
%\textbf{define}  H            $\leftarrow$ set of structure-specific information defined by LEAF(C) for each leaf\\
%
%\textbf{update} W in childrens of root R with the results of \textsc{WeightEstimation}(C,LEAF(C),H)\\
%
%\textbf{for each} node R' in CHILD(R) that is not a leaf\\
%
%$\triangleright$\textsc{TreeWeightEstimation}(R', W)\\
%\hrule
%
%\end{minipage}
%
%\\\\\\Descripton algo2\\
%The algorithm used estimates the global prior of the leaves of the current node, based on the number of pixel which belong to the child classes.
%This number of pixels is calculated from the segmentation computed in the histogram.(CF ..)\\\\
%%
%\begin{minipage}{1\textwidth}
%
%\hrule
%\textbf{\\Algorithm 2:} \textsc{WeightEstimation}(C,LEAF(C), W,H)%
%\hrule
%variables definition
%\textbf{\\define}  T $\leftarrow$ set of total weight of leaves in LEAF(C). Leaves weights are contained in H\\ 
%\textbf{define}  E      $\leftarrow$ set of weight for each node of C\\ 
%\textbf{for each}  node of C\\
%$\triangleright$E=E+H : Get the total weight of each node\\
%
%W=E/T\\
%
%\textbf{return} (W)\\
%\hrule

%\end{minipage}
%\\\\\\

  \begin{figure}\centering
  \includegraphics[width=0.7\textwidth]{Images/Screenshots/GlobalPrior.png}
  \caption{Tool developped for the global prior weights estimation}\label{fig:globalpriors}
  \end{figure}

The new tool is presented in figure (\ref{fig:globalpriors}) (A). When the user moves a cursor, it changes the intensity range for each class. The tool provides a real time feedback to the user, about about what he is doing. The user sees figure \ref{fig:globalpriors}) (B) which is updated in real time, regarding the position of all the cursors consequence. Clicking on "update tree" (in figure \ref{fig:globalpriors}) (A)), the estimated size of a class is computed and the information is stored into the tree structure (parameter $H$).\\
\par
We presented the contribution we did to the EM semgentation module in Slicer 3. It lookds usefull, nevertheless we still don't know if it will have a real impact on the whole segmentation process. In the next chapter, we present the influence of the contributions on the whole segmentation proocess.

%As we can see, the results obtained are good and accurate.The time of processing is fast and most of the people are now able to understand these parameters, even they are not familiar with EMS. This is an intuitive way to get an estimation of the GPW parameters.\\
%Of course some sfsdfs must be done.
%If there is a strong bias in the images we process, the pixels of a same class will have different values. Then, a segmentation based on intensity will provide bad results. Moreover, if the user wants to segment 2 classes which have the same color, just using the spatial information for example, they won't be able to use it properly.Thus, the user must always keep in mind that it is just an estimation and that he has to check if the values are accurate.
%


\chapter{Results and discussion}\label{sec:results}
This chapter will present the importance of the contribution we did, regarding the final segmentation. We will not discuss about the enhanced usability of the module. It will allow you to see how the segmentation has changed. We will then discuss about the next contribution which could be brought to enhance the segmentation and the usability of a such module. 

%
\section{Results}
%
Here we get going with a presentation of different results, using the different contributions. The results obtained will then be reviewed by a specialist to evaluate the segmentation. We work with the same dataset for all the segmentation. We chose a biased one to show the importance of the bias correction.
%
\subsection{Original segmentation}
We first present the segmentation obtained without any contribution. From these results, it will show the importance of the contributions done. We will first describe the testing process. Then we will present the results of the segmentation and finally, an expert will estimate it.
\subsubsection{Testing process}
Thanks to Sonia Pujol, a tutorial for the EM segmentation in Slicer 3 is available at \url{http://www.na-mic.org/Wiki/images/2/2f/AutomaticSegmentation_SoniaPujol_Munich2008.ppt}. We followed it with a biased image. The atlases we used for the segmentation are the one available at \url{http://www.na-mic.org/Wiki/images/b/b7/AutomaticSegmentation.tar.gz}.

\subsubsection{Results}
The results of the segmentation are presented in figure (\ref{fig:NC}).
  \begin{figure}\centering
  \includegraphics[width=0.95\textwidth]{Images/Screenshots/NC.png}
  \caption{Results of the segmentation with bias correction}\label{fig:NC}
  \end{figure}
  
  %
  \par
We can explain the results in many different ways. First of all, the bias field was not corrected (see section \ref{biasfieldcorrectionregistration}). That means that the registration was poor . Moreover, tissues' distribution was bad. The means values were not accurate and the variances were too big. The initialization was bad and lead to a bad registration. In addition to the problem of intensities inhomogeinities during the tissues' distributions' initialization, we did a manual sampling. As presented in section \ref{sec:CDS}, it can have importance during the segmentation.  
  
\subsubsection{Expert's point of view}
It is clearly bad and we didn't really need an expert to see that the segmentation is not good. From the expert point of view, the grey matter is over estimated. This is exactly what we saw in figure (\ref{fig:biasedunbiased}). The grey matter (light blue) is overestimated in (B), the class distribution through the biased volume.

\subsection{Bias corrected segmentation}
%
Here we present the results of a segmentation using the intensities inhomogeinities correction. We first describe the testing process. Then we present and interpret the results obtained. Finally an expert evaluates the results.
%
\subsubsection{Testing process}
We followed the same tutorial we used in the previous section. We proceeded exactly the same way. The only difference is that we now correct the image intensities inhomogeinities before the segmentation.

%
\subsubsection{Results}
The results of the final segmentation are presented in figure (\ref{fig:C_M_L}). (A) presents the original unbiased image, (B) the segmented image, using a manual sampling. Visually, the results appears better but as long as we are not experts, an expert will estimate the result of the segmentation in the next section.
%

  \begin{figure}\centering
  \includegraphics[width=0.95\textwidth]{Images/Screenshots/C_M_L.png}
  \caption{Results of the segmentation with label map}\label{fig:C_M_L}
  \end{figure}

%
%More images with a better resolution are in annexes (), to really see the importance of the contribution.
%
\par
We can explain the difference. It is due to a better registration and a better class distribution. We showed in section (\ref{biasfieldcorrectionregistration}) that the registration was a lot improved. Moreover, this is not the only avantage. Indeed, correcting the bias field, the classes to be segmented have a better distribution. We  demonstrated it in figure (\ref{fig:biasedunbiased}). The tissues have a better distribution, especially gray matter (light blue) in this case.
%
\subsubsection{Expert's point of view}
Some troubles appear in the skull's segmentation. The segmentation returns that there is grey matter and white matter in the bones.A bad label is attribuated to a tissue after the segmentation. It is called misclassification. It can be attribuated the the partial volume artifacts. The partial volume artifact effect is caused by imaging voxels containing two different tissues (skull and air in our case). The voxel's returning intensity will be an average of the two tissue present in the voxel. Moreover the skull is not perfectly segmented because sometimes it appears that there is big holes in the skull with air inside, which is not accurate. The misclassification in the other areas are mostly due to the partial volume artifacts.
%
\subsection{Label map sampling segmentation}
The segmentation  will now let us evaluate the influence of the sampling method on the segmentation. We first describe the testing process. Then we present and interpret the results obtained. Finally an expert evaluates the results.
%
\subsubsection{Testing process}
We proceeded exactly the same way as we did during the bias corrected segmentation. The only difference is that now, we used a labelmap to initialize the classes distributions.
%
\subsubsection{Results}

The results of the final segmentation are presented in figure (\ref{fig:NC_C_L}). (A) presents the original corrected image, (B) the result of the segmentation. Visually, we can't prononce ourselves and an expert will estimate the result of the segmentation in the next section. This time, the classes are supposed to have a better distribution than in the previous segmentation, as we saw in figure (\ref{fig:intensitynormalization}). HOLES FILLED, CLASS MORE HOMOGENEOUS. More screenshots with a better resolution are presented in App.~\ref{app:results} to have a better understanding of the differences between manual and labelmap sampling.

  \begin{figure}\centering
  \includegraphics[width=0.95\textwidth]{Images/Screenshots/NC_C_L.png}
  \caption{Results of the segmentation with bias correction}\label{fig:NC_C_L}
  \end{figure}
 
%
\subsubsection{Expert's point of view}
The results are pretty good too. The skull is better segmented. Nevertheless, in cerebulum, the segmentation differs from the previous one. It is underestimated  with the labelmap sampling whereas before it was over estimated. The cerebulum is a very specific area of the brain because there is a lot of white and grey matter in a small area. The segmentation is hard and one time again, the results must due to partial volume effects. According to our expert is a complex task. We can't easily segment it if we segment the whole brain. To segment the cerebrulum properly, we should only segment the region of interest.
%

%
%

%

\section{Future work}
The results are good. Nevertheless, a lot of work has still to be done in order to enhance the segmentation process. The main things to be done will be presented now.
\par
Here we get going with the correction of the bias field in the MR image. The results are good but a major issue appeared: the time of processing to correct the bias. On a a 255x255x255 volume, it typically last more than thirty minutes. This is not acceptable and we have to go through the algorithm to have a deep understanding of the bias correction. It will be then possible to adapt the ITK filter to our problem. The time of processing can then be a lot reduced, according to Tustison.
\par
Regarding the global prior weights estimation, a problem appears. When the image to be segmented contains some classes which have a close distribution, the tool is no longer effective. Indeed the estimation is only based on the intensity of the voxels. A usefull contribution could consist in representing each class as a gaussian, instead of a finite portion of the histogram. The histogram would be the result of the summation of all the gaussians. Assuming the each class is a gaussian is not a bad assumption since we us this definition in the EM algorithm.
\par
These are the main improvement to be done in the next months.




\include{conclusion}
%-----------------------------------------------------------
\addcontentsline{toc}{chapter}{\numberline{}Bibliography}
\begin{thebibliography}{9999}%\enlargethispage{\baselineskip}
%
\bibitem[1]{1}A.P. Dempster, N.M. Laird, and D.B. Rubin, "Maximum likelihood from incomplete data via the em algorithm", \textsl{Journal of the Royal Statistical Society: Series B}, vol. 39, pp. 1-38, November 1977.
%
\bibitem[2]{2}Y. Weiss, "Bayesian motion estimation and segmentation", \textsl{PhD thesis}, Massachussets Intitute of Technology, May 1998.
%
\bibitem[3]{3}R.C. Hardie, K.J. Barnard, and E.E. Armstrong, "Joint MAP registration and high-resolution image estimation using a sequence undersampled images", \textsl{IEEE Transaction on Image Processing}, vol. 6, pp. 1621-1633, December 1997.
%
\bibitem[4]{4}M. Murgasova, "Tutorial on Expectation-Maximization: Application to Segmentation of Brain MRI", May 2007.
%
\bibitem[5]{5}S. Borman, "The Expectation Maximization Algorithm",January 2009.
%
\bibitem[6]{6}Wikipedia, "Expectation-Maximization algorithm", \url{http://en.wikipedia.org/wiki/Expectation-maximization_algorithm}, June 2009.
%
\bibitem[7]{7}G. McLachlan, and T. Krishnan, "The EM Algorithm and Extensions", \textsl{John Wiley \& Sons}, New York, 1996.
%
\bibitem[8]{8}K. V. Leemput, F. Maes, D. Vandermeulen et al., "Automated model-based tissue classification of MR images of the brain", \textsl{IEEE Transaction on Medical Imaging}, 18(10), pp. 857-908, 1999.
%
\bibitem[9]{9}K. V. Leemput, F. Maes, D. Vandermeulen et al., "Automated model-based bias field correction of MR images of the brain", \textsl{IEEE Transaction on Medical Imaging} 18(10), pp. 885-896, 1999.
%
\bibitem[10]{10}W. M. Wells III, W.E.L. Grimson, R. Kikinis et al., "Adaptative segmentation of MRI data", \textsl{IEEE Transaction on Medical Imaging} 15(4), pp. 429-442, 1996.
%
\bibitem[11]{11}K.M. Pohl et al., "A Hierarchical Algorithm for MR Brain Image Parcellation", \textsl{IEEE Transaction on Medical Imaging} 26(9), pp. 1201-1212, 2007.
%
\bibitem[12]{12}D. Hoa, and A. Micheau, "e-MRI, Magnetic Resonance Imaging physics and technique course on the web", \textsl{http://e-mri.org}, 2007.
%
\bibitem[13]{13}Kitware, "Insight Toolkit", \url{http://www.itk.org/}, 2009.
%
\bibitem[14]{14}N. Tustison, "Nick's N3 ITK Implementation For MRI Bias Field Correction", \url{http://www.insight-journal.org/browse/publication/640}, June 2009.
%
\bibitem[15]{15}Wikipedia, "Multivariate Normal Distribution",\\ \url{http://en.wikipedia.org/wiki/Multivariate_normal_distribution}, June 2009.
%
\bibitem[16]{16}Wikipedia, "Correlation", \url{http://en.wikipedia.org/wiki/Correlation}, June 2009.
%
\bibitem[17]{17}Wikipedia, "Covariance matrix", \url{http://en.wikipedia.org/wiki/Covariance$\_$matrix}, June 2009.
%
\bibitem[18]{18}K.M. Pohl et al., "Automatic Segmentation Using Non-Rigid Registration", \textsl{MICCAI}, 2005.
%
\bibitem[19]{19}J.G. Sled, A.P. Zijdenbos, and A.C. Evans, "A non parametric method for automatic correction of intensity nonuniformity in mri data", \textsl{IEEE Transactions on Medical Imaging} 17(1), pp. 87-97, February 1998.
%
\bibitem[20]{20}Wikipedia, "Full width at half maximum", \url{http://en.wikipedia.org/wiki/Full_width_at_half_maximum}, July 2009.
%
\bibitem[21]{21}B. Zitova, and J. Flusser, "Image registration methods: a survey", \textsl{Image and vision computing} 21, pp. 977-1000, 2003.
%
\bibitem[22]{22}S. Pieper,M. Halle, and R. Kikinis, "3D SLICER", \textsl{Proceedings of the 1st IEEE International Symposium on Biomedical Imaging} 1, pp. 632-635, 2004.
%
\end{thebibliography}
%\vfill
%\begin{flushright}\small Prepared in \LaTeXe\ by RCJ\end{flushright}

%-----------------------------------------------------------
\appendix
%\include{app4}
\chapter{Statistics}\label{app:formulas} Here we present the main formulas we used in this reports and some fundamentals about probabilities.


\section{Fundamentals}\label{f:Fundamentals}

$P(A)$ is the probability that $A$ is realized.\\
\par
$P(A|B)$ is the probability that $A$ is realized, knowing $B$.\\
\par
$P(A,B)$ is the probability that $A$ and $B$ are realized at the same time.\\
\par
$P(A|B) = \frac{P(A,B)}{P(B)}$



\section{Bayes' theorem}\label{f:Bayes}
\subsection{Theorem}
Let $S$ be a sample of space. If $A_1,A_2,...A_n$ are mutually exclusive and exhaustive events such as $P(A_i)\neq 0$ for all $i$.Then for any event $A$ which is a subset of $S = A1\cup A_2 \cup ... \cup A_n$ and $P(A) > 0$ we have
\begin{equation*}
P(A_i|A)= \frac{P(A_i)P(A|A_i)}{\sum_{j=1}^n P(A_j)P(A|A_j)}
\end{equation*}

\subsection{Proof}
We have $S = A1\cup A_2 \cup ... \cup A_n$ and $A_i \cap A_j = \varnothing$ for $i \neq j$. Since $A \subseteq S$

\begin{align*}
\Rightarrow A &= A \cap S\\
              &= A \cap (A1\cup A_2 \cup ... \cup A_n)\\
              &= (A \cap A_1)\cup(A \cap A_2)\cup ... \cup(A \cap A_n)
\end{align*}
Moreover
\begin{equation*}
P(A \cap A_i) = P(A)P(A_i|A)
\end{equation*}

So

\begin{align*}
 P(A) &= P(A\cap A_1) + P(A\cap A_2) + ... + P(A\cap A_n)\\
      &= P(A)P(A_1|A) + P(A)P(A_2|A) + ... + P(A)P(A_n|A)
\end{align*}

And

\begin{equation*}
P(A|A_i) = \frac{P(A \cap A_i}{P(A)}
\end{equation*}

Finally we obtain
\begin{equation*}
P(A_i|A)= \frac{P(A_i)P(A|A_i)}{\sum_{j=1}^n P(A_j)P(A|A_j)}
\end{equation*}

\section{Jensen's inequality}\label{f:Jensen}
\subsection{Inequality}
Let $f$ be a convex function defined on an interval $I$. If $x_1,x_2,...,x_n \in I$ and $\lambda_1, \lambda_2, ... , \lambda_n \geq 0$ with $\sum_{i=1}^n \lambda_i = 1$,

  \begin{equation*}
  f(\sum_{i=1}^n \lambda_i x_i) \leq \sum_{i=1}^n \lambda_i f(x_i)
  \end{equation*}

\subsection{Proof}
To show that the theorem is true we proceed by induction.\\
\begin{itemize}
\item \textbf{Initialization}\\
This is trivial for $n=1$.\\
\item \textbf{Hypothesis at rank $n$} 

 \begin{equation*}
  f(\sum_{i=1}^n \lambda_i x_i \leq \sum_{i=1}^n \lambda_i f(x_i))\\
  \end{equation*}

\item \textbf{Demonstration at rank $n+1$}

 \begin{align*}
  f(\sum_{i=1}^{n+1} \lambda_i x_i ) &= f(\lambda_{n+1} x_{n+1} + \sum_{i=1}^n \lambda_i x_i)\\
                                     &= f(\lambda_{n+1} x_{n+1} + (1-\lambda_{n+1})\frac{1}{1-\lambda_{n+1}}\sum_{i=1}^n \lambda_i x_i)\\
                                     &\leq \lambda_{n+1} f(x_{n+1}) + (1-\lambda_{n+1})f(\frac{1}{1-\lambda_{n+1}}\sum_{i=1}^n \lambda_i x_i)\\
                                     &= \lambda_{n+1} f(x_{n+1}) + (1-\lambda_{n+1})f(\sum_{i=1}^n \frac{\lambda_i}{1-\lambda_{n+1}} x_i)\\
                                     &\leq \lambda_{n+1} f(x_{n+1}) + (1-\lambda_{n+1})\sum_{i=1}^n \frac{\lambda_i}{1-\lambda_{n+1}} f(x_i)\\
                                     &= \lambda_{n+1} f(x_{n+1}) + \sum_{i=1}^n \lambda_i f(x_i)\\
                                     &= \sum_{i=1}^{n+1} \lambda_i f(x_i)\\      
  \end{align*}
\end{itemize}

With a concave function (in opposition to convex), the inequality becomes:

 \begin{equation*}
  f(\sum_{i=1}^n \lambda_i x_i) \geq \sum_{i=1}^n \lambda_i f(x_i)
  \end{equation*}
%\chapter{Definitions}\label{app:definitions} We definie some terms, to enhance the understanding of the reader.

\section{Smooth function}\label{f:smooth}

%\include{app5}
\chapter{Results of the segmentation}\label{app:results}
Here we present the results of segmentations in the case of manual and labelmap sampling. The left image represents the results of the segmentation after a manual sampling and the other image represents the results after a labelmap sampling.

%\section{Axial view}

  \begin{figure}[htb]\centering
  \includegraphics[width=1\textwidth]{Images/Screenshots/axialcomp.png}
  \caption{Axial view of the segmentation with different sampling methods}
  \end{figure}

%\section{Coronal view}

  \begin{figure}[htb]\centering
  \includegraphics[width=1\textwidth]{Images/Screenshots/coronalcomp.png}
  \caption{Coronal view of the segmentation with different sampling methods}
  \end{figure}

%\section{Sagittal view}

  \begin{figure}[htb]\centering
  \includegraphics[width=1\textwidth]{Images/Screenshots/sagittalcomp.png}
  \caption{Sagittal view of the segmentation with different sampling methods}\label{fig:NC_C_L}
  \end{figure}
%-----------------------------------------------------------
\end{document}
