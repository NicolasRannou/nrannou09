% This text is proprietary.
% It's a part of presentation made by myself.
% It may not used commercial.
% The noncommercial use such as private and study is free
% Sep. 2005 
% Author: Sascha Frank 
% University Freiburg 
% www.informatik.uni-freiburg.de/~frank/
%
% additional usepackage{beamerthemeshadow} is used
%  
%  \beamersetuncovermixins{\opaqueness<1>{25}}{\opaqueness<2->{15}}
%  with this the elements which were coming soon were only hinted
\documentclass{beamer}
\usepackage{beamerthemeshadow, pgf}
\usepackage{graphicx}
\usepackage{multimedia}


\begin{document}

\title{Nouveau processus de segmentation dans Slicer 3}  
\author{Nicolas Rannou}
%\mylogo{Images/Logos/logo_ISEN.jpg}
%\date{\today} 
\date[] % (optional, should be abbreviation of conference name)
{}
\institute[ISEN] % (optional, but mostly needed)
{Institut Sup\'erieur de l'\'Electronique et du Num\'erique}
%\begin{titlepage}

%\pgfputat{\pgfxy(0,-6.5)}{\pgfbox[right,base]{\pgfimage[width=1cm]{Images/Logos/logo_spl.jpg}}}
%\begin{center}
%\includegraphics[width=.2\textwidth]{Images/Logos/logo_spl.jpg}
%\end{center}

%\end{titlepage}

%\logo{\includegraphics[width=2cm]{Images/Logos/logo_spl.jpg}}

%\pgfdeclareimage[height=0.85cm]{logo-gauche}{Images/Logos/logo_spl.jpg}
%\pgfdeclareimage[height=0.85cm]{logo-droite}{Images/Logos/logo_spl.jpg}
%\logo{\pgfuseimage{logo-droite}}

\pgfdeclareimage[height=0.75cm]{Images/Logos/logo_spl.jpg}{Images/Logos/logo_spl.jpg}
\logo{\pgfuseimage{Images/Logos/logo_spl.jpg}}


%\setbeamertemplate{header left}
%{
%\logo{\pgfuseimage{logo-gauche}}
%\vfill%
%\rlap{\hskip0.1cm\insertlogo}%
%\vskip15pt%
%}

\expandafter\def\expandafter\insertshorttitle\expandafter{%
  \insertshorttitle\hfill\insertframenumber\,/\,\inserttotalframenumber}

%\setbeamertemplate{footline}[page number]

%\setbeamertemplate{footline}{\hfill\insertframenumber/\inserttotalframenumber} 

\frame{\titlepage} 


\section{Introduction} 
\frame{\frametitle{Introduction} 

  \begin{columns}
  \begin{column}{0.3\textwidth}
    \includegraphics[width=4cm]{Images/intro.png}
  \end{column}
  \begin{column}{0.7\textwidth}
  
  \begin{block}{Contexte} 
  
    \begin{itemize}[<+->]
      \item IRM c\'er\'ebrale
      \item Nombre important de donn\'ees
      \item Segmentation manuelle co\^uteuse en temps
      \item Variabilit\'e intra- et inter-expert
      \item D\'eveloppement de m\'ethodes de segmentation automatiques des tissus
      \item Apparition de la segmentation par exceptation-maximisation
    \end{itemize}

  \end{block}    
  
  \end{column}
  \end{columns}

  
%\begin{itemize}
%\item{\textbf{Logo:} [mylogo] use one of the norwegian versions of the logo on the titlepage, your logo, or no logo at all}
%\end{itemize}
}


\frame{\frametitle{Introduction} 

  \begin{columns}
  \begin{column}{0.3\textwidth}
    \includegraphics[width=4cm]{Images/intro.png}
  \end{column}
  \begin{column}{0.7\textwidth}
  
   \begin{alertblock}{Probl\`eme} 
   Peu utilis\'e car
    \begin{itemize}[<+->]      
      \item processus de segmentation doit \^etre am\'elior\'e
      \item param\`etres optimums durs \`a choisirs
      \item param\`etres peu explicites
    \end{itemize}

  \end{alertblock}    
  
   %\begin{block}{Objectif} 

%    \begin{itemize}[<+->]      
%      \item Am\'eliorer le processus de segmentation (correction du biais)
%      \item Faciliter le processus de segmentation (Fournir des outils \`a l'utili)
%    \end{itemize}

%  \end{block}    
  
  \end{column}
  \end{columns}

}




\section{Segmentation par expectation maximisation}
\frame{\frametitle{Plan}\tableofcontents[currentsection]}  
\subsection{Principe}
\frame{\frametitle{Segmentation par expectation maximisation}Principe} 
\frame{ \frametitle{La segmentation}
  \begin{columns}
  \begin{column}{0.3\textwidth}
    \includegraphics[width=4cm]{Images/Atlas.jpg}
  \end{column}
  \begin{column}{0.7\textwidth}
  
   \begin{block}{D\'efinition} 
   Diviser un ensemble en parties d\'elimit\'ees

   \end{block}    
  
  
  \end{column}
  \end{columns}
}
\frame{ \frametitle{Origine de la segmentation par expectation-maximisation}
\begin{block}{} 
    \begin{itemize}[<+->]      
      \item En 1977, Dempster, Laird et Rubin ont g\'en\'eralis\'e un principe utilis\'e depuis longtemps par les auteurs
      \item Utilis\'e pour r\'esoudre des probl\`emes de classifications o\`u des donn\'ees sont manquantes
    \end{itemize}

  \end{block}  
}
\frame{ \frametitle{Principe de la segmentation par expectation-maximisation}
Deux \'etapes, l'expectation et la maximisation. \\
Soit $\Phi$, un set contenant les param\`etres \`a estimer (moyenne et variance pour chaque tissu).\\
$\Phi$ est initialis\'e par l'utilisateur.

\begin{block}{\'Etape d'expectation} 
    \begin{itemize}[<+->]      
      \item Estime la probabilit\'e que le set de param\`etres soit bon
    \end{itemize}

  \end{block}  
  
  \begin{block}{\'Etape de maximisation} 
    \begin{itemize}[<+->]      
      \item Estime un nouveau set de param\`etres
    \end{itemize}

  \end{block}  

}

\frame{ \frametitle{R\'esum\'e de la segmentation par expectation-maximisation}
\begin{center}
 \includegraphics[width=3.5cm]{Images/Graphics/EMSimple.png}
 \end{center}
}

\subsection{EM segmentation dans Slicer 3}
\frame{\frametitle{Segmentation par expectation maximisation}EM segmentation dans Slicer 3} 
\frame{ \frametitle{EM segmentation dans Slicer 3}
  \begin{block}{Informations suppl\'ementaires} 
    \begin{itemize}[<+->]      
      \item Atlas probabilistes
      \item Segmentation multi-canaux
      \item Correction des inhomog\'einit\'es de l'intensit\'e
      \item Information hi\'erarchique
    \end{itemize}

  \end{block} 
    

  \begin{center}
 \includegraphics<1>[width=4cm]{Images/Screenshots/atlasT1.png}

 \includegraphics<2>[width=4cm]{Images/Screenshots/T1ForLabel.png}
 \includegraphics<2>[width=4cm]{Images/Screenshots/T2ForLabel.png}
 
 \includegraphics<3>[width=4cm]{Images/Screenshots/T1SagittalNotCorrected.png}
 \includegraphics<3>[width=4cm]{Images/Screenshots/T1SagittalCorrected.png}
  
 \includegraphics<4>[width=4cm]{Images/Graphics/treeStructure.png}
 \end{center} 
  
  
}
\frame{ \frametitle{Processus de segmentation dans Slicer 3}  
  \begin{center}
 \includegraphics<1>[width=10cm]{Images/Graphics/previousalgo.png}
 \end{center} 
}



\section{Contributions} 
\frame{\frametitle{Plan}\tableofcontents[currentsection]} 
\subsection{Initialisation des tissus \`a segmenter}
\frame{\frametitle{Contributions}Initialisation des tissus \`a segmenter} 
\frame{\frametitle{Initialisation des tissus \`a segmenter}
  \begin{alertblock}{Pr\'esentation du probl\`eme} 
  M\'ethodes actuelles d'initialisation
    \begin{itemize}[<+->]      
      \item Manuelle : dur \`a estimer
      \item Semi-automatique : peu repr\'esentatif du tissu et non reproductible
    \end{itemize}

  \end{alertblock} 
  }
  
 \frame{\frametitle{Initialisation des tissus \`a segmenter}
    \begin{block}{Solution propos\'ee} 

    \begin{itemize}[<+->]  
      \item Initialisation \`a l'aide d'une "labelmap"    
      \item Repr\'esentatif du tissu \`a segmenter
      \item Reproductible
    \end{itemize}
  \end{block}
  
         \begin{center}
 \includegraphics[width=10cm]{Images/Screenshots/labelmaps.png}
 \end{center} 
}

 \frame{\frametitle{Initialisation des tissus \`a segmenter}
    \begin{exampleblock}{\'Evaluation des r\'esultats} 
    Comparaison Semi-automatique/Labelmap (mati\`ere blanche, IRM T1)
    \begin{itemize}[<+->]  
      \item Semi-automatique (10 \'echantillons) : $\mu = 543$, $\sigma = 1105$ 
      \item Labelmap ($\simeq$ 200 \'echantillons) : $\mu = 489$, $\sigma = 592$ 
    \end{itemize}
  \end{exampleblock}
  
}

\subsection{\'Evaluation de la s\'election des tissus}
\frame{\frametitle{Contributions}\'Evaluation de la s\'election des tissus} 
\frame{\frametitle{\'Evaluation de la s\'election des tissus}
  \begin{alertblock}{Pr\'esentation du probl\`eme} 
   Aucun moyen de savoir si l'initialisation est la meilleure possible
  \end{alertblock} 
  }
  
 \frame{\frametitle{\'Evaluation de la s\'election des tissus}
    \begin{block}{Solution propos\'ee} 

    \begin{itemize}[<+->]  
      \item Repr\'esentation de la distribution des tissus sous forme de Gaussiennes    
      \item Connaissant les tissus \`a segmenter, on peut en d\'eduire si l'initialisation est bonne
    \end{itemize}
  \end{block}
  
         \begin{center}
 \includegraphics[width=3.5cm]{Images/Screenshots/gauss_label_2.png}
 \end{center} 
}

 \frame{\frametitle{\'Evaluation de la s\'election des tissus}
    \begin{exampleblock}{\'Evaluation des r\'esultats} 
    \begin{itemize}[<+->]  
      \item Os repr\'esent\'e en bleu  
      \item Gauche : Semi-automatique $\|$ Droite : Labelmap
    \end{itemize}
  \end{exampleblock}
  
           \begin{center}
 \includegraphics[width=10cm]{Images/Screenshots/classdistribution.png}
 \end{center} 
  
}

\subsection{Correction des inhomog\'einit\'es d'intensit\'e}
\frame{\frametitle{Contributions}Correction des inhomog\'einit\'es d'intensit\'e} 
\frame{\frametitle{Correction des inhomog\'einit\'es d'intensit\'e}
  \begin{alertblock}{Pr\'esentation du probl\`eme} 
   \begin{itemize}[<+->]  
      \item Processus de segmentation fait pour traiter les IRM 
      \item Inhomog\'einit\'es d'intensit\'e probl\`eme r\'ecurrent
      \item Probl\`eme trait\'e tardivement
      \item Apparition de probl\`emes de recalage et de distribution
    \end{itemize}
  \end{alertblock} 
  
\begin{center}
 \includegraphics<1>[width=5cm]{Images/Screenshots/T1SagittalNotCorrected.png}
 \includegraphics<2>[width=5cm]{Images/Screenshots/T1SagittalNotCorrected.png}
 \includegraphics<3>[width=10cm]{Images/Graphics/previousalgo.png}
 \includegraphics<4>[width=10cm]{Images/Graphics/previousalgo.png}
 \end{center}   
  
  }
  
 \frame{\frametitle{Correction des inhomog\'einit\'es d'intensit\'e}
    \begin{block}{Solution propos\'ee} 

    \begin{itemize}[<+->]  
      \item Nouveau processus de segmentation
      \item Pour am\'eliorer recalage
      \item Pour am\'eliorer la distribution des tissus
    \end{itemize}
  \end{block}
  
         \begin{center}
 \includegraphics[width=10cm]{Images/Graphics/newalgo.png}
 \end{center} 
}

 \frame{\frametitle{Correction des inhomog\'einit\'es d'intensit\'e}
    \begin{exampleblock}{\'Evaluation des r\'esultats} 
    Recalage\\
    Histogrammes joints sans correction du bias
    \end{exampleblock}
  
           \begin{center}
 \includegraphics[width=10cm]{Images/Screenshots/jointhistograms1.png}
 \end{center} 
  
}

 \frame{\frametitle{Correction des inhomog\'einit\'es d'intensit\'e}
    \begin{exampleblock}{\'Evaluation des r\'esultats} 
    Recalage\\
    Histogrammes joints apr\`es correction du bias
    \end{exampleblock}
  
           \begin{center}
 \includegraphics[width=10cm]{Images/Screenshots/jointhistograms2.png}
 \end{center} 
  
}

 \frame{\frametitle{Correction des inhomog\'einit\'es d'intensit\'e}
    \begin{exampleblock}{\'Evaluation des r\'esultats} 
    Distribution des tissus
    \end{exampleblock}
  
    \begin{center}
    \includegraphics[width=10cm]{Images/Screenshots/biasedunbiased.png}
    \end{center} 
  
}

\subsection{\'Evaluation du param\`etre de normalisation}
\frame{\frametitle{Contributions}\'Evaluation du param\`etre de normalisation} 
\frame{\frametitle{\'Evaluation du param\`etre de normalisation}
  \begin{alertblock}{Pr\'esentation du probl\`eme} 
   Difficile \`a \'evaluer pr\'ecis\'ement
  \end{alertblock} 
  
  \begin{block}{Solution propos\'ee} 

   D\'eveloppement d'un outil d'\'evaluation
  \end{block}
  
         \begin{center}
 \includegraphics[width=4.5cm]{Images/Screenshots/intensityNormalization.png}
 \end{center} 
}

% \frame{\frametitle{Correction des inhomog\'einit\'es d'intensit\'e}
%    \begin{exampleblock}{\'Evaluation des r\'esultats} 
%    Recalage\\
%    Histogrammes joints sans correction du bias
%    \end{exampleblock}
%  
%           \begin{center}
% \includegraphics[width=10cm]{Images/Screenshots/jointhistograms1.png}
% \end{center} 
%  
%}
%\subsection{\'Evaluation des param\`etres hi\'erarchiques}
%\frame{\frametitle{numbered lists with pause}}



\section{R\'esultats} 
\frame{\frametitle{Plan}\tableofcontents[currentsection]} 
\subsection{Segmentation sans contribution}
\frame{\frametitle{R\'esultats}Segmentation sans contributions} 
\frame{\frametitle{Segmentation sans contributions}

\begin{block}{M\'ethode de tests} 
  \begin{itemize}[<+->]  
      \item Segmentation multi-canal
      \item Suit un tutoriel
      \item Images cibles contiennent des inhomog\'einit\'es d'intensit\'e
    \end{itemize}
  \end{block}

}

\frame{\frametitle{Segmentation sans contributions}

  \begin{columns}
  \begin{column}{0.4\textwidth}
    \includegraphics[width=5cm]{Images/Screenshots/EMS_MANUAL_SAGITTAL.png}
  \end{column}
  \begin{column}{0.5\textwidth}
  
  
  \begin{block}{Point de vue de l'expert} 
  
    \begin{itemize}[<+->]
      \item Mati\`ere grise surestim\'ee
      \item Mauvaise segmentation
      \item Inutilisable
    \end{itemize}

  \end{block}    
  
  \begin{block}{Discussion} 
  
    \begin{itemize}[<+->]
      \item Mauvais recalage
      \item Mauvaise distribution ds tissus
    \end{itemize}

  \end{block}    
  
  \end{column}
  \end{columns}

}

\subsection{Segmentation apr\`es correction des inhomog\'einit\'es d'intensit\'e}
\frame{\frametitle{R\'esultats}Segmentation apr\`es correction des inhomog\'einit\'es d'intensit\'e} 
\frame{\frametitle{Segmentation apr\`es correction des inhomog\'einit\'es d'intensit\'e}

\begin{block}{M\'ethode de tests} 

\begin{itemize}[<+->]  
      \item Segmentation multi-canal
      \item Suit un tutoriel
      \item Images cibles contiennent des inhomog\'einit\'es d'intensit\'e
      \item Nouveau processus de segmentation
    \end{itemize}

  \end{block}
  
          \begin{center}
 \includegraphics<4>[width=10cm]{Images/Graphics/newalgo.png}
 \end{center} 

}

\frame{\frametitle{Segmentation apr\`es correction des inhomog\'einit\'es d'intensit\'e}

  \begin{columns}
  \begin{column}{0.4\textwidth}
    \includegraphics[width=5cm]{Images/Screenshots/EMS_C_MANUAL_SAGITTAL.png}
  \end{column}
  \begin{column}{0.5\textwidth}
  
  
  \begin{block}{Point de vue de l'expert} 
  
    \begin{itemize}[<+->]
      \item Erreurs de classification dans la zone de l'os
      \item Effet de volume partiel
    \end{itemize}

  \end{block}    
  
  \begin{block}{Discussion} 
  
    \begin{itemize}[<+->]
      \item Segementation meilleure
      \item Probl\`eme li\'es \`a la segmentation de l'os peuvent \^etre r\'esolus
    \end{itemize}

  \end{block}    
  
  \end{column}
  \end{columns}

}

\subsection{Segmentation avec la nouvelle m\'ethode d'initialisation des tissus}

\frame{\frametitle{R\'esultats}Segmentation avec la nouvelle m\'ethode d'initialisation des tissus} 
\frame{\frametitle{Segmentation avec la nouvelle m\'ethode d'initialisation des tissus}

\begin{block}{M\'ethode de tests} 
\begin{itemize}[<+->]  
      \item Segmentation multi-canal
      \item Suit un tutoriel
      \item Images cibles contiennent des inhomog\'einit\'es d'intensit\'e
      \item Nouveau processus de segmentation
      \item Initialisation des tissus par Labelmap
    \end{itemize}

  \end{block}

}

\frame{\frametitle{Segmentation avec la nouvelle m\'ethode d'initialisation des tissus}

  \begin{columns}
  \begin{column}{0.4\textwidth}
    \includegraphics[width=5cm]{Images/Screenshots/EMS_C_LABEL_SAGITTAL.png}
  \end{column}
  \begin{column}{0.5\textwidth}
  
  
  \begin{block}{Point de vue de l'expert} 
  
    \begin{itemize}[<+->]
      \item Os mieux segment\'e
      \item Sous estimation de la mati\`ere blanche dans le cerebulum
    \end{itemize}

  \end{block}    
  
  \begin{block}{Discussion} 
  
    \begin{itemize}[<+->]
      \item Meilleure distribution des tissus, et notament de l'os
      \item Segmentation du cerebulum, probl\`eme complexe
    \end{itemize}

  \end{block}    
  
  \end{column}
  \end{columns}

}



\section{Perspectives}
\frame{\frametitle{Plan}\tableofcontents[currentsection]} 
\frame{\frametitle{Perspectives}Perspectives} 
\frame{\frametitle{Perspectives}

  \begin{block}{Priorit\'es} 
  
    \begin{itemize}[<+->]
      \item Faire des efforts au niveau des algorithmes
      \item Am\'eliorer la vitesse de l'agorithme d'EM segmenation
      \item Am\'eliorer la vitesse de l'algorithme de correction du biais
    \end{itemize}

  \end{block}   

}

\section{Conclusion}
\frame{\frametitle{Conclusion}

  \begin{block}{} 
  
    \begin{itemize}[<+->]
      \item Pr\'esentation d'un nouveau processus de segmentation
      \item R\'esultats int\'eressants
      \item Vitesse de traitement doit \^etre am\'elior\'ee
    \end{itemize}

  \end{block}  

}

\frame{\frametitle{Conclusion}
Merci de votre attention ! \\
\bigskip
Des questions ? 
}

\end{document}

