%-----------------------------------------------------------
%Abstract
\section*{Abstract}
\vspace{0.2cm}
\par
The use of systems to manage large amounts of data becomes increasingly important in the field of neuroscience as researchers work on neuroimaging methods such as positron emission tomography (TEP) and functional magnetic resonance imaging (fMRI). Indeed, these techniques have produced an explosion of new findings in human neuroscience associated with an explosion of the amount and the size of the data involved in the researches. Moreover, the necessity to share both data and results of processing with others researchers and laboratories is very important nowadays in order to create efficient collaborations and thus to help to make new findings.
\par
In this context, this document propose an interesting solution through the creation of the Neurobiological Image Management System (NIMS) allowing the management of neurological data. Thus, this project consists on an experimental software platform designed to ease and to improve the sharing, storage, mining and analysis of data coming from scanners in different groups of research. This has been made in a way that can evolve depending on the needs of each thanks to an open-source and flexible approach. The document presents similar projects existing in other research centers, a first version with a standard neurobiological imaging workflow and gives some ideas about the creation of an ontology for a such system.
\vspace{0.3cm}\\
\underline{Keywords:} data sharing, database, neurobiological images, open-source, metadata, workflow, ontology
\vspace{0.5cm}\\
\section*{R\'esum\'e}
\vspace{0.2cm}
\par
L'utilisation de syst\`emes pour prendre en charge d'importantes quantit\'es de donn\'ees devient de plus en plus fr\'equente dans le domaine des neurosciences alors que les chercheurs travaillent sur des m\'ethodes de neuroimagerie telles que la tomographie par \'emission de positions (TEP) et l'imagerie par r\'esonance magn\'etique fonctionnelle (IRMf). En effet, ces techniques ont produit une explosion de nouvelles d\'ecouvertes en neuroscience associ\'ee \`a une explosion de la quantit\'e et de la taille des donn\'ees impliqu\'ees dans les recherches. De plus, la n\'ecessit\'e de partager \`a la fois les donn\'ees et les r\'esultats des traitements avec les autres (chercheurs et laboratoires) est tr\`es importante de nos jours afin de cr\'eer des collaborations efficaces et ainsi d'aider \`a faire de nouvelles d\'ecouvertes.
\par
Dans ce contexte, ce document propose une solution int\'eressante \`a travers la cr\'eation d'un syst\`eme de management d'images neurobiologiques. Ce projet consiste en une plateforme logicielle exp\'erimentale construite pour faciliter et am\'eliorer le partage, le stockage, l'extraction et l'analyse de donn\'ees provenant de scanners dans diff\'erents groupes de recherche. Ceci a \'et\'e fait d'une mani\`ere qui peut \'evoluer en fonction des besoins de chacun gr\^ace \`a une approche open-source et flexible. Le document pr\'esente des projets similaires dans d'autres centres de recherches, ainsi qu'une premi\`ere version de celui-ci avec un flux de travail (workflow) standardis\'e et donne quelques id\'ees concernant la cr\'eation d'une ontologie pour un tel syst\`eme.
\vspace{0.3cm}\\
\underline{Mots cl\'es :} partage de donn\'ees, base de donn\'ees, images neurobiologiques, open-source, metadonn\'ees, flux de travail, ontologie