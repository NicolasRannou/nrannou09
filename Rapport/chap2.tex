\chapter{Expectation-maximization applied to brain segmentation}\label{sec:EM}
%Each main chapter or section should start with a short description of
%what it holds, and why. Top tip --- begin the whole writing enterprise
%with a first draft of this little bit for each chapter. It will force
%you to think about overall structure.\\
%dsfdfs
%\par
%
Here we get going with theory of the expectation-maximization, applied to brain segmentation and show firstly a simple approach of the problem. Then there's a
more realistic approach of the problem with different constraints, followed by a presentation of the algorithm used in Slicer 3\footnote{open source software developped in the SPL for biomedical engineering purpose}.
%
\section{Presentation of the EM segmentation}
%Magnetic resonance imaging (MRI) is very well suited for analyzing human soft tissue anatomy. It provides high resolution high resolution 3D volumetric data with high resolution between soft tissues. Nevertheless, this technique has some disavantages. Indeed MRI images can be alterated by some artifacts as movement, magnetic suceptibility, http://www.e-mri.org, aliasing, etc.. Another main problem is the apparition of a bias on MRIs. This bias results from QDSFQSDF. Correcting this problem is very important in the purpose of image processing. If we don't, the same tissue will have different intensities through the volume, which can bree mistakes during the segmentation process.
The EM algorithm was explained in 1977 by Arthur Dempster, Nan Laird, and Donald Rubin\cite{1}. They generalized and developped a method used in several times by authors, for particular applications. It is widely used to solve problems where data aisre "missing".
The EM algorithm is an iterative algorithm which works in two steps: Expectation and Maximization. It can be use to solve a lot of image processing's problems like classification, restoration\cite{3}, motion estimation\cite{2}, etc.. 
Since the generalization of the algorithm, a lot of related papers were proposed. Most of them bring algorithms derived from the original one to adapt it to particuliar problems using additional informations.
Nowadays, EM algorithms are become a popular tool for classification problems.  It is particulary well suited for brain MR images segmentation.
A lot of algorithms exist. They present complex frameworks using spatial information, neighborhood or intensity inhomogeneities to enhance the classification.\\
In the SPL, the algorithm developped uses spatial, structural and intensity inhomogeneities informations to segment the brain. 
%

\section{Fundamentals}
you should read this part if..
\subsection{Statistical model used for the brain}
\subsection{Gaussian mixture model}
\subsection{Maximum likelihood model}
Here's an inline formula: \(E=mc^2\), and here's the same
thing displayed:\[E=mc^2.\]A matrix needs to be displayed:
\[  \det\left(\begin{array}{cc}
        a & b \\  c & d \\
      \end{array}\right) = ad-bc,   \]
and here's a result that's displayed and labelled:
\begin{equation}\label{eq:display}
\lim_{a\to\infty}\int_0^a\exp(-x^2)\,dx=\fr12\sqrt\pi.
\end{equation}Note the punctuation in eq.~(\ref{eq:display}) \etc,
which recognises that displayed equations are parts of sentences.
\par
Note also that a smaller (\verb+\textstyle+) size of numerical fraction
is often useful in a displayed equation. The appropriate
\verb+\newcommand+ called \verb+\fr+ is defined in the
\Quote{preamble}, in file \texttt{report.tex}.
\par
More complicated limits on \verb+\int+ (and \verb+\sum+) need to
be enclosed in braces $\{\cdots\}$.
\par
In an inline formula, avoid ugly fractions such as \(v=\frac st\) and
\(\frac{dy}{dx}=x+y\). It looks much better to write \(v=s/t\) and
\(dy/dx=x+y\). That is, keep \verb+\frac+ for
display:\[\frac{dy}{dx}=x+y\]where it belongs.
\par
Beware of inadvertent blank lines in your \texttt{.tex} file; they often
creep in immediately after a displayed equation. A blank line gives a
new paragraph, usually indented, which may not be what you want. You can 
avoid blank lines altogether by breaking paragraphs with \verb+\par+
instead, as done here.
%
\section{Expectation maximization algorithm}
first simplified then real
\subsubsection{Simplified algorithm}
\subsubsection{Expectation maximization algorithm}
Now an equation on several lines using \verb+eqnarray+:
\begin{eqnarray*}
  |\vec A|^2 &=& a_1^2+a_2^2  \\
             &=& \sin^2\theta+\cos^2\theta \\
             &=& 1
\end{eqnarray*}--- where the star on \verb+eqnarray*+ suppresses
all line-numbering\footnote{And re-phrase material where any displayed
equation splits between pages.}.
\par
Again, with just one line numbered for reference:
\begin{eqnarray}
  |\vec A|^2 &=& a_1^2+a_2^2  \label{line-one}\\
             &=& \sin^2\theta+\cos^2\theta \nonumber \\
             &=& 1  \nonumber
\end{eqnarray}--- where \verb+\nonumber+ suppresses numbering
otherwise. Now you can refer to equation~(\ref{line-one}).
\par
If you want to number the equations in Chapter 2 as (2.1), (2.2), etc,
then\footnote{Top tip: easily find out about any \LaTeX\ command by
putting it \textit{with} its backslash (and perhaps the word \lq latex')
into your favourite search engine.} put \lq\verb+\numberwithin+' into
\textsl{Google} or see \cite[Sec 8.2.14]{MG}.
\par
Notice that in equations you use \verb+\lim+, \verb+\exp+, \verb+\sin+
and \verb+\cos+ and not plain $lim$, $exp$, $sin$ and $cos$. See
\cite[Sec.~3.3]{NSS} for a list.
\par
For a lot of complicated multi-line formulas it's better to use the more
sophisticated display environments provided by the package
\texttt{amsmath} --- see Sec.~8.2 of \comp\ \cite{MG}.
For instance the \texttt{cases} environment is used to get
\[\theta(x)=\begin{cases}0&\text{if $x<0$,}\\
		1&\text{if $x>0$.}\end{cases}\]
And if you need continued fractions, instead of 
\[x=\frac{1}{a_2+\frac{1}{a_3+\frac{1}{a_4+\cdots}}},\]a better-looking
result
is\[x=\frac{\strut1}{\displaystyle a_2+\frac{\strut1}{\displaystyle
a_3+\frac{\strut1}{\displaystyle a_4+\cdots}}},\]for which
\texttt{amsmath}
provides the command \verb+\cfrac+ \cite[Sec.~8.4.2]{MG}.
\par
The package \texttt{amssymb} \cite[Chap.~8]{MG} extends the range of
mathematical symbols (\eg\(\mathbb{R}\), \(\mathbb{Z}\) and
\(\mathbb{N}\) are available).
\par
And its sister \texttt{amstext} provides the command \verb+\text+ to
put words into a displayed equation \[ \text{like this: }E=mc^2. \]Note
that you need to put in by hand the spacing between the text and the
mathematical symbols.
%
\section{Expectation maximization algorithm used in Slicer 3}\label{angels}
In this part, we present the algorithm which has been intergated in Slicer 3, and discuss of the limitations of this one. Some informations has been added to the algorithm to get the best and most automatic segmentation as possible.
%
\subsection{Spatial information}
%
\subsection{Structure information}
%
\subsection{Intensity inhomogeneities information}
%
\subsubsection{Discussion}
%This is theorem~\ref{angels}.
%\begin{description}
%\item[Theorem]: The whole is greater than the sum of its parts.
%\item[Proof:] See App.~\ref{app:proofs}, Sec.~\ref{pf:angels}.
%\end{description}
%For numbered theorems, use \verb+\newtheorem+ \cite[Sec.~3.8]{NSS}.
%\par
%Here's the main sequence of theorems.
%\newtheorem{Main}{Theorem} % this could go into the preamble
%\begin{Main}[my first result]
%The first theorem, showing
%\[ \bar D^n\quad\text{and}\quad\bar{D^n} \]
%where the first bar is over just the \lq D\rq\ and the second is over
%the whole expression. Also the word \lq and\rq\ is inside displayed
%maths, using \verb+\text+ with extra space before and after.
%\end{Main}And that's how to do left and right quotation marks, too,
%with \verb+\lq+ and \verb+\rq+.
%\begin{Main}
%The second theorem --- with \verb+\overline+ inline: $\overline{D^n}$.
%\end{Main}
%\begin{Main} the third theorem\end{Main}
%Here's another sequence of results.
%\newtheorem{second}{Lemma} % this too could go into the preamble
%\begin{second}[my second result]
%The next result (a mere lemma) \dots\end{second}
%\begin{second}\dots\ and the next\end{second}
%\comp\ \cite[Sec.~3.3.3]{MG} describes how to use the
%packages \texttt{amsmath} and \texttt{amsthm} to state (and
%cross-reference) theorems, lemmas, definitions, proofs, \etc\ in a more 
%sophisticated way.
%
\section{Nothing}
Each chapter should end with a round-up of its contents and a link
with the contents of the next.
\par
After formulas, equations and theorems, the next important
topics are graphics and tables.
