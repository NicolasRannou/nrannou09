\chapter{Results and discussion}\label{sec:results}
This chapter will present the importance of the contribution we did, regarding the final segmentation. We will not discuss about the enhanced usability of the module. It will allow you to see how the segmentation has changed. We will then discuss about the next contribution which could be brought to enhance the segmentation and the usability of a such module. 

%
\section{Results}
%
Here we get going with a presentation of different results, using the different main contributions. We will conpare results obtained with the previous workflow and the new one. The results obtained will then be reviewed by a specialist to evaluate the segmentation. We always work with the same dataset. We chose intentionally a biased one.
%
\subsection{Bias correction}
%
Here we present the utility of the intensities inhomogeinities correction. It is usefull for the registration
%
\subsubsection{Testing process}
A tutorial for EM segmentation in Slicer 3 is available at SDFS. We followed it, one time with a biased image, one time with a correct image. All the parameters remains the same in the segmentation workflow.

%
\subsubsection{Results}correction
The results of the final segmentation are presented in figure (\ref{fig:NC_C_L}). (A) presents the original biased image, (B) the corrected image, (C) the result of the segmentation with the biased image and (D) the result of the segmentation with the corrected image. The reults appears better but as long as we are not experts, an expert will estimate the result of the segmentation in the next section.
%
  \begin{figure}\centering
  \includegraphics[width=0.55\textwidth]{Images/Screenshots/NC_C_L.png}
  \caption{Results of the segmentation with bias correction}\label{fig:NC_C_L}
  \end{figure}


%
\par
%more comments ttttttttttttttttttttt ttttttttttttt ttttttttttt tttttttttttt ttttttttttt ttttttttttt tttttttttttttt ttttttttttttt tttttttttttttt tttttttttttt ttttttttttttttt\\
%
\par
Nevertheless, we can explain the difference. It is due to a better registration and a better class distribution. QSDFQ
More images with better resolution in annexes SFDSDF.

%
\subsubsection{Experts's point of view}
%
FFFFFFFFFFFFFFFFFFFFFFFFFFFF\\FFFFFFFFFFFFFFFFFF\\FFFFFFFFFFFFFFFFFFF\\FFFFFFFFFFFFFFFFFFFFFFFFFF\\FFFFFFFFFFFFFFFFFFFF\\FFFFFFFFFFFFFFFFFFFFFFFFFFFFFFFFFF
%
\subsection{Class Selection}
compare results regarding accuracy of distribution
%
\subsubsection{Testing process}
We followed the same tutorial as in the previous section. We used each time a corrected image
%
\subsubsection{Results}

  \begin{figure}\centering
  \includegraphics[width=0.55\textwidth]{Images/Screenshots/C_M_L.png}
  \caption{Results of the segmentation with label map}\label{fig:C_M_L}
  \end{figure}
  
underestimation
overestimation
holes filled

%
\subsubsection{Expert's point of view}
%
VVVVVVVVVVVVVVVVVVVVVVVVVVVVVV

\section{Future work}

PRIOR
A major issue of this approach is that when the image is biased and when some class have a close distribution. It is no longer effective because it is only based on the intensity of the voxels. Integrate the gaussian distribution in the estimation process.

\newpage
\section*{Acknowledgements}
\addcontentsline{toc}{chapter}{\numberline{}Acknowledgements}
Ron Kikinis who gave me the opportunity to carry out my intersnship in the SPL.
Sylvain Jaume who supervises me during all my work.
Steve  helped a lot

