\chapter{Results and discussion}\label{sec:results}
This chapter will present the importance of the contribution we did, regarding the final segmentation. We will not discuss about the enhanced usability of the module. It will allow you to see how the segmentation has changed. We will then discuss about the next contribution which could be brought to enhance the segmentation and the usability of a such module. 

%
\section{Results}
%
Here we get going with a presentation of different results, using the different contributions. The results obtained will then be reviewed by a specialist to evaluate the segmentation. We work with the same dataset for all the segmentation. We chose a biased one to show the importance of the bias correction.
%
\subsection{Original segmentation}
We first present the segmentation obtained without any contribution. From these results, it will show the importance of the contributions done. We will first describe the testing process. Then we will present the results of the segmentation and finally, an expert will estimate it.
\subsubsection{Testing process}
Thanks to Sonia Pujol, a tutorial for the EM segmentation in Slicer 3 is available at \url{http://www.na-mic.org/Wiki/images/2/2f/AutomaticSegmentation_SoniaPujol_Munich2008.ppt}. We followed it with a biased image. The atlases we used for the segmentation are the one available at \url{http://www.na-mic.org/Wiki/images/b/b7/AutomaticSegmentation.tar.gz}.

\subsubsection{Results}
The results of the segmentation are presented in figure (\ref{fig:NC}).
  \begin{figure}\centering
  \includegraphics[width=0.95\textwidth]{Images/Screenshots/NC.png}
  \caption{Results of the segmentation with bias correction}\label{fig:NC}
  \end{figure}
  
  %
  \par
We can explain the results in many different ways. First of all, the bias field was not corrected (see section \ref{biasfieldcorrectionregistration}). That means that the registration was poor . Moreover, tissues' distribution was bad. The means values were not accurate and the variances were too big. The initialization was bad and lead to a bad registration. In addition to the problem of intensities inhomogeinities during the tissues' distributions' initialization, we did a manual sampling. As presented in section \ref{sec:CDS}, it can have importance during the segmentation.  
  
\subsubsection{Expert's point of view}
It is clearly bad and we didn't really need an expert to see that the segmentation is not good. From the expert point of view, the grey matter is over estimated. This is exactly what we saw in figure (\ref{fig:biasedunbiased}). The grey matter (light blue) is overestimated in (B), the class distribution through the biased volume.

\subsection{Bias corrected segmentation}
%
Here we present the results of a segmentation using the intensities inhomogeinities correction. We first describe the testing process. Then we present and interpret the results obtained. Finally an expert evaluates the results.
%
\subsubsection{Testing process}
We followed the same tutorial we used in the previous section. We proceeded exactly the same way. The only difference is that we now correct the image intensities inhomogeinities before the segmentation.

%
\subsubsection{Results}
The results of the final segmentation are presented in figure (\ref{fig:C_M_L}). (A) presents the original unbiased image, (B) the segmented image, using a manual sampling. Visually, the results appears better but as long as we are not experts, an expert will estimate the result of the segmentation in the next section.
%

  \begin{figure}\centering
  \includegraphics[width=0.95\textwidth]{Images/Screenshots/C_M_L.png}
  \caption{Results of the segmentation with label map}\label{fig:C_M_L}
  \end{figure}

%
%More images with a better resolution are in annexes (), to really see the importance of the contribution.
%
\par
We can explain the difference. It is due to a better registration and a better class distribution. We showed in section (\ref{biasfieldcorrectionregistration}) that the registration was a lot improved. Moreover, this is not the only avantage. Indeed, correcting the bias field, the classes to be segmented have a better distribution. We  demonstrated it in figure (\ref{fig:biasedunbiased}). The tissues have a better distribution, especially gray matter (light blue) in this case.
%
\subsubsection{Expert's point of view}
Some troubles appear in the skull's segmentation. The segmentation returns that there is grey matter and white matter in the bones.A bad label is attribuated to a tissue after the segmentation. It is called misclassification. It can be attribuated the the partial volume artifacts. The partial volume artifact effect is caused by imaging voxels containing two different tissues (skull and air in our case). The voxel's returning intensity will be an average of the two tissue present in the voxel. Moreover the skull is not perfectly segmented because sometimes it appears that there is big holes in the skull with air inside, which is not accurate. The misclassification in the other areas are mostly due to the partial volume artifacts.
%
\subsection{Label map sampling segmentation}
The segmentation  will now let us evaluate the influence of the sampling method on the segmentation. We first describe the testing process. Then we present and interpret the results obtained. Finally an expert evaluates the results.
%
\subsubsection{Testing process}
We proceeded exactly the same way as we did during the bias corrected segmentation. The only difference is that now, we used a labelmap to initialize the classes distributions.
%
\subsubsection{Results}

The results of the final segmentation are presented in figure (\ref{fig:NC_C_L}). (A) presents the original corrected image, (B) the result of the segmentation. Visually, we can't prononce ourselves and an expert will estimate the result of the segmentation in the next section. This time, the classes are supposed to have a better distribution than in the previous segmentation, as we saw in figure (\ref{fig:intensitynormalization}). HOLES FILLED, CLASS MORE HOMOGENEOUS. More screenshots with a better resolution are presented in App.~\ref{app:results} to have a better understanding of the differences between manual and labelmap sampling.

  \begin{figure}\centering
  \includegraphics[width=0.95\textwidth]{Images/Screenshots/NC_C_L.png}
  \caption{Results of the segmentation with bias correction}\label{fig:NC_C_L}
  \end{figure}
 
%
\subsubsection{Expert's point of view}
The results are pretty good too. The skull is better segmented. Nevertheless, in cerebulum, the segmentation differs from the previous one. It is underestimated  with the labelmap sampling whereas before it was over estimated. The cerebulum is a very specific area of the brain because there is a lot of white and grey matter in a small area. The segmentation is hard and one time again, the results must due to partial volume effects. According to our expert is a complex task. We can't easily segment it if we segment the whole brain. To segment the cerebrulum properly, we should only segment the region of interest.
%

%
%

%

\section{Future work}
The results are good. Nevertheless, a lot of work has still to be done in order to enhance the segmentation process. The main things to be done will be presented now.
\par
Here we get going with the correction of the bias field in the MR image. The results are good but a major issue appeared: the time of processing to correct the bias. On a a 255x255x255 volume, it typically last more than thirty minutes. This is not acceptable and we have to go through the algorithm to have a deep understanding of the bias correction. It will be then possible to adapt the ITK filter to our problem. The time of processing can then be a lot reduced, according to Tustison.
\par
Regarding the global prior weights estimation, a problem appears. When the image to be segmented contains some classes which have a close distribution, the tool is no longer effective. Indeed the estimation is only based on the intensity of the voxels. A usefull contribution could consist in representing each class as a gaussian, instead of a finite portion of the histogram. The histogram would be the result of the summation of all the gaussians. Assuming the each class is a gaussian is not a bad assumption since we us this definition in the EM algorithm.
\par
These are the main improvement to be done in the next months.



