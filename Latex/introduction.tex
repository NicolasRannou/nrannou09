%-----------------------------------------------------------
%Introduction
\section*{Introduction}
\vspace{0.2cm}
\par
My training period took place in the Vision, Imaging Sciences and Technology Activities (VISTA) laboratory, one of the research centers of the Stanford University in Palo Alto. During approximately 5 months, from April 6th  to August 31th, I worked in collaboration with an outside company, ConjureLLC represented by Jorge PHILLIPS and Anthony SHERBONDY. Brian WANDELL and Bob DOUGHERTY, part of the laboratory were the supervisors of the project by checking the evolution of the work.
\par
Nowadays, the increasingly complex research questions addressed by neuroimaging research impose substantial demands on computational infrastructures. These infrastructures need to support management of massive amounts of data in a way that affords rapid and precise data analysis, to allow collaborative research, and to achieve these aims securely and with minimum management overhead. This is particularly true in the field of functional magnetic imaging (fMRI), which has produced an explosion of the amount and the size of data involved. For example, a single research study may require the repeated processing, using computationally demanding and complex applications, of thousand of files corresponding to hundreds of functional MRI studies.
\par
Thus, the main goal of the project is to create an open-source and flexible management prototype system for neurobiological images in order to improve their sharing, storage, mining and analysis between several researchers groups. This is designed to improve researcher and team productivity by supporting neurobiological imaging workflow through management of images and data classification and storage.  This prototype should be sufficiently modular and extensible to be able to accommodate near future changes in functionality, implementation technologies and deployment modalities, and scalable to enterprise quality in future projects.
\par
In this report, the system shown is a result of the work made on the project during the 5 months I was in the laboratory and is due to be used in a first step by several laboratories of the psychology department of Stanford University. First of all, after a quick presentation of Stanford, of the Vista laboratory and of the overall project, the first part will be concentrated on the similar projects existing in other research centers as the Computational Neuroscience Applications Research Infrastructure (CNARI) or the Extensible Neuroimaging Archive Toolkit (XNAT). Afterward, a second part will explain the technological choices made for our system, the standard workflow used for the data coming from a scanner and the structure of the overall system. In a third part, I will expose some ideas extracted from the literature to create an ontology for our system. Finally, I will present the next steps that the team will achieve in the future to answer to the initial requirements of the project.
